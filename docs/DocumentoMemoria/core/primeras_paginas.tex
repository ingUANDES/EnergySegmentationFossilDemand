% Paginas de titulo, derechos de autor y otras mas.

% Jaime Cisternas, julio 2004, agosto 2010

% declaracion especifica de pdfLaTeX
% permite incluir figuras con \includegraphics[...]{figurename}
\DeclareGraphicsExtensions{.jpg,.pdf,.mps,.png}

%%%%%%%%%%%%%%%%%%%%%%%%%%%%%%%%%%%%%%%%%%%%%%%%%%%%%%%%%%%%%%%%

  \pagenumbering{roman}
% \setcounter{page}{1}

% Escribe la pagina de titulo, incluyendo el autor y la facultad

  \thispagestyle{empty}
  \textsc{
  \vspace*{0cm}
  \begin{center}
    \Large
     Universidad de los Andes \\
     Facultad de Ingeniería y Ciencias Aplicadas\\
  \end{center}
     \vspace{1cm}
  \begin{center}
     \includegraphics[angle=0,height=5cm]{logos/Logo-UANDES.png}
  \end{center}
     \vspace{1cm}
  \begin{center}
     \Large
     \titulo
  \end{center}
  \vspace{0.5cm}
  \begin{center}
    \Large
    \nombreautor
  \end{center}
  \vspace{0.5 cm}
  \begin{center}
    Memoria para optar al título de \\
    Ingeniero Civil Industrial 
    Mención Gestión de Operaciones\\
    \vspace{1cm}
    Profesor Guía: \nombreprofuno \\
    \vspace{0.5cm}
    \codigo \\
    \vspace{0.5cm}
    Santiago, \mes\ de \anio
  \end{center}
  }

%%%%%%%%%%%%%%%%%%%%%%%%%%%%%%%%%%%%%%%%%%

% Escribe la segunda pagina de titulo para ser firmada por la comision

\cleardoublepage
\thispagestyle{empty}

\begin{center}

\vspace*{2cm}
\parbox{10cm}{
\noindent
Certifico que he leído esta memoria y que en mi opinión
su alcance y calidad son completamente adecuados como para ser considerada
una memoria conducente al título de Ingeniero.
\vspace{1cm}

\hfill
\begin{tabular}{c}
\hspace{8cm} \\
\hline
\nombreprofuno \\
(Profesor Guía)
\end{tabular}

\vspace*{1.5cm}

}

\end{center}

%%%%%%%%%%%%%%%%%%%%%%%%%%%%%%%%%%%%%%%%%%

% Espaciado entre lineas
  \onehalfspacing
  %\doublespacing

% Hace pagina de derechos de autor

  \cleardoublepage
  \thispagestyle{empty}
  %\vspace*{0in}
  \begin{center}
    \copyright\ \nombreautor\ \anio \\
    Todos los derechos reservados.
  \end{center}

%%%%%%%%%%%%%%%%%%%%%%%%%%%%%%%%%%%%%%%%%%%%%%%%%%%%%%%%%%%%%%%%%%%%%%

% Genera agradecimientos leyendo definicion de agradecimientos

  \cleardoublepage \addcontentsline{toc}{section}{Agradecimientos}
  \begin{center} \Large \textbf{Agradecimientos} \end{center}

Un primer agradecimiento se lo dedico a mi familia, por su constante apoyo tanto emocional como económico a lo largo de toda mi carrera universitaria. Agradezco la motivación y deseos de mejorar que me traspasaron durante todos estos años.
\vspace{2.5mm}

Un segundo agradecimiento es para el profesor Sebastián Cea que me ofreció la oportunidad de desarrollar la memoria con él y trabajar juntos en la mejora de su modelo. Su apoyo constante, reuniones continuas y el perfil personalizado con el cual me guió durante el último año, me permitió aprender y avanzar de forma ordenada en mi memoria. 
\vspace{2.5mm}

Un tercer agradecimiento se lo dedico a Felipe Feijoo por el taller que realizó sobre la resolución de problemas de equilibrio y su ayuda en el área computacional de este trabajo.
\vspace{2.5mm}

Finalmente, un agradecimiento a Orlando Contreras, tutor del Centro de Escritura, quien me apoyó con la redacción de este documento. Su ayuda tuvo un rol fundamental en el orden de mis ideas y el de este documento para realizar la mejor escritura posible.


% Genera resumen leyendo definicion de resumen


  \cleardoublepage
  \refstepcounter{dummy} \addcontentsline{toc}{section}{Resumen}
  \begin{center} \Large \textbf{Resumen} \end{center}
La presente memoria estudia la incorporación de atención racional en el modelo de \textit{cap and trade} de dos etapas creado por \citeB{amigo_two_2021}. Este modelo es una alternativa para fiscalizar las emisiones de dióxido de carbono de las empresas generadoras de electricidad, en que un subastador (también llamado planificador social) emite permisos que luego vende a las empresas generadoras y con los cuales ellas se limitan a contaminar.
\vspace{2.5mm}

La atención limitada guarda relación con la incapacidad de procesar toda la información disponible en la toma de decisiones. La inclusión de atención racional en el modelo se realiza porque la fijación del límite de emisiones implica un costo de adquirir o estimar esa información. La incorporación de este concepto es dificultoso en este modelo ya que este es un problema de optimización de capacidad con varios agentes y restricciones de equilibrio. Con el fin de resolver el problema, se deben transformar varios problemas en uno solo del tipo \textit{Mixed Complementarity Problem} (MCP, por sus siglas en inglés). La dificultad implícita es que el problema resultante es altamente no lineal.
\vspace{2.5mm}

En esta memoria se explica qué son los modelos de equilibrio, las condiciones de optimalidad, los MCP, se proporcionan ejemplos de cómo resolverlos en distintos lenguajes de programación y se replica el modelo original. Por último, se realiza la implementación de atención racional.
\vspace{2.5mm}

La implementación se realiza en el modelo del subastador. Se realizan dos versiones: \textit{Profit oriented}, en la cuál el subastador experimenta pérdidas según el rendimiento determinado y otra \emph{Welfare oriented} enfocada en el bienestar social (sobre la cual se realizaron dos formulaciones). Esta última optimiza el beneficio social en lugar de las utilidades económicas. 
\vspace{2.5mm}

Se resolvieron los nuevos modelos y se compararon con el original. Por un lado, los resultados del modelo \textit{Profit oriented} evidencian el impacto significativo del rendimiento en las variables óptimas del subastador. Este produjo, los precios de electricidad más bajos entre todos los modelos.
\vspace{2.5mm}

Por otro lado, el modelo Tasa Cuadrada (primera formulación \emph{Welfare oriented}) emite la menor cantidad de permisos para altos presupuestos de carbono. El Modelo con Precisión (segunda formulación \emph{Welfare oriented}) proporciona, junto al \textit{Profit oriented}, los precios de permisos de emisión más bajos para todos los presupuestos.
\vspace{2.5mm}

De acuerdo con lo anterior, se concluye que la propuesta aumenta el bienestar social y que existe potencial para perfeccionar ambas versiones y proponer una mejora al sistema.




% Genera dedicatoria

 % \cleardoublepage \vspace*{1.5in}
  %\begin{flushright} \dedicatoria %\end{flushright}

% Produce indice general, listas de figuras y de tablas

  \cleardoublepage
  \tableofcontents
  \cleardoublepage
  \listoffigures
  \cleardoublepage
  \listoftables
  \cleardoublepage

  \normalsize
  \pagenumbering{arabic}

%%%%%%%%%%%%%%%%%%%%%%%%%%%%%%%%%%%%%%%%%%%%%%%%%%%%%%%%%%%
