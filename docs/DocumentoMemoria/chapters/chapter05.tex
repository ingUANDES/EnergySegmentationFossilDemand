\chapter{Conclusiones y Recomendaciones}
\label{c5} % la etiqueta para referencias

UNA DE LAS REOMENDACIONES DE COMO MEJORAR EL MODELO ES CONSIDERARE EL EFECTO O LAS POSIBILIDADES DE LOS DISTINTOS ESCENARIOS DE DEMANDA AL CREAR EL MODELO DEL SUBASTADOR.


Calibrar modelo para determinar adecuadamentes los precios de venta de permisos en el mercado de permisos


El estudio del cálculo en Chile para la determinación de su presupuesto de carbono y porque no existe una referencia clara a como se cálculo este valor en cada NDC.


Agregar costos de operacion del subastador para incorporar un papel de regulador de emisiones en el futuro.

Estudiar con más profundidad el costo marginal de atencion racional

estudiar una distribucion de probabilidad para el error de los permisos y agregarla al problema.


Investigar como chile podría disminuir el error de calculo de CAP, para así incluirlo como el mecanismo o encontrar el verdadero costo marginal que eliminaria el ruido (disminuiria el error)

