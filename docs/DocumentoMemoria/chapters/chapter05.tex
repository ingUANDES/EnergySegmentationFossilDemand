\chapter{Conclusiones y Recomendaciones}
\label{c5} % la etiqueta para referencias

Los resultados muestran que existe potencial para incorporar el concepto de atención racional en el modelo de \textit{cap and trade} de dos etapas, con el fin de aumentar el bienestar social. La comparación entre los modelos, realizada en el capítulo anterior, evidencia que los modelos nuevos entregan menor cantidad de permisos, menor precio de venta de estos y de electricidad en varios presupuestos de carbono establecidos por el gobierno.
\vspace{2.5mm}

Con el análisis anterior, se concluye que se cumplió el objetivo general de este trabajo. Se reformuló el sistema de \textit{cap and trade} de dos etapas creado por \citeB{amigo_two_2021} al incluir el concepto de atención racional satisfactoriamente, entregar resultados coherentes y con pruebas de potencialidad para crear un modelo reproducible en la realidad.
\vspace{2.5mm}

Sumado a lo anterior, se cumplieron todos los objetivos específicos y, adicionalmente, se incorporó y cumplió el objetivo de traducir el modelo al lenguaje de programación Julia. Esta traducción permitirá agilizar nuevas mejoras en el modelo. 
\vspace{2.5mm}

El concepto de atención racional se incorporó al subastador ya que este adquiere el rol de planificador social en el sistema. Él es quien, finalmente, será responsable de las emisiones generadas por las empresas eléctricas en el agregadp. Por lo tanto, tiene la capacidad de determinar la cantidad de permisos emitidos que mayor bienestar social produzcan.
\vspace{2.5mm}

Las dos versiones del modelo, \textit{Profit Oriented} y \emph{Welfare oriented} (modelo Tasa cuadrada y Precisión), demuestran que si el subastador incorpora la atención racional en su perfil de administrador,  disminuirá las emisiones de permisos en comparación con las propuestas por el gobierno (CAP), en especial mientras mayor sea el CAP. 
\vspace{2.5mm}

En el modelo original del subastador, este emite la cantidad de permisos que el CAP permite, sin tener un papel real en el modelo. Esta primera aproximación de incorporación de atención racional, no solo cambia su perfil, sino que también afecta sus utilidades al incorporar un gasto de inversión en información. Una recomendación para mejorar los modelos es que al subastador se le incorporen gastos de operación asociados con la fiscalización de las empresas emisoras y que este tenga un papel de compra de permisos y, tal vez, emisión de nuevos permisos en el largo plazo, si el escenario lo amerita.
\vspace{2.5mm}

Los trabajos realizados por \citeB{andres_new_2014} y \citeB{ballantyne_audit_2015} muestran que existe un error asociado con el cálculo del presupuesto de carbono establecido por Chile. Demuestran que este error se repite en la mayoría de los países, pero con algunas diferencias en su magnitud.
\vspace{2.5mm}

Debido a lo anterior, se realiza la recomendación principal de este trabajo: mejorar los modelos propuestos focalizando la atención racional en cómo los errores de cálculo de presupuestos pueden ser disminuidos. Con esto se debe, en primer lugar, entender de que forma se puede disminuir este error y qué costos implica. En segundo lugar, incorporarlo al modelo del subastador y formular el modelo en que la inversión en información genere un menor error. Finalmente, en esta formulación,  el subastador evaluará cuánto invertir en información, el número de permisos por emitir y el bienestar social como resultado de estas decisiones.




