% cap2.tex

\chapter{MARCO TEÓRICO}
\label{c2} 


\section{\textit{Cap and Trade} y otros sistemas actuales}\label{c22}
%Agregar desde segundo párrafo de la Revision bibliográfica del hito 1

En el contexto de la generación de energía, el medio ambiente es directamente afectado por la tecnologías utilizadas. Para generar energía se necesita de la combustión de un material, del aprovechamiento mecánico de una energía natural externa u otros como plantas nucleares. Estos sistemas provocan, en distintos niveles y formas, efectos en el medio ambiente. Particularmente las termoeléctricas son consideradas como las que más emisiones de dióxido de carbono producen.
\vspace{2.5mm}

La última importante mención por parte del Gobierno de Chile sobre su plan para combatir esta situación es la propuesta climática a largo plazo \citeB{gobierno_de_chile_estrategia_2021} que Chile presentó en la COP26 el mes de Octubre de 2021. En ella, Chile propone medidas drásticas para combatir las emisiones realizando una propuesta específica en la sección de generación de energía. En ella se presenta un presupuesto límite de emisión de carbono para los generadores de energía. Pero, en contraste, no se incluye un mecanismo o plan específico de como las empresas generadoras lograrán esto y como se llevará a cabo la supervisión de esta meta.
\vspace{2.5mm}

Desde que se reconoció al efecto invernadero como un problema, varios países han comenzado a intervenir, directa o indirectamente, en las industrias que más aportan al efecto. 
\vspace{2.5mm}

Por una parte, el impuesto al carbono o impuesto a emisiones de $CO_2$ consiste en un costo adicional para la generación de electricidad con energías no renovables. El impuesto es cobrado por cada tonelada de este compuesto emitido. Finlandia, en 1990, fue el primer país en implementarlo y desde ahí 18 otros países del continente y muchos otros en el planeta lo fueron implementando a distintos valores por tonelada emitida \citeB{asen_carbon_2021}.
\vspace{2.5mm}

Por otro lado, existen los modelos de \textit{cap and trade}. Concepto creado por Thomas Crocker, consiste en que un ente regulador determina capacidades máximas de contaminación por emisiones a las empresas de la industria eléctrica, donde los emisores son regulados y se les asignan permisos donde si estos superan la capacidad permitida, deben pagar penalizaciones \citeB{hanemann_cap-and-trade_2010}. 
\vspace{2.5mm}

En un contexto donde se busca cumplir con promesas por país para lograr el acuerdo de París (no aumentar la temperatura global en más de 2°C para fines del siglo), un modelo de \textit{cap and trade} es el que parece ser una mejor opción. Esto se debe a que permite establecer, inicialmente, capacidades límites por industria o por país. Entonces, la búsqueda de lograr contaminar un determinado máximo es más creíble que sea realizado con un modelo que regule el contaminante de cada aportante estableciendo una capacidad máxima para ellos. El problema es que cada tipo de \textit{cap and trade} implementado puede afectar en distintas formas la industria, las empresas involucradas y la demanda energética. Por un lado, se puede lograr minimizar emisiones gracias a la reducción en la generación y por otro lado se puede lograr aumentar la cantidad de empresas generadoras con energías renovables.
\vspace{2.5mm}

\citeB{chen_renewable_2021} prueba que aplicar un mecanismo de \textit{cap and trade} es mucho más efectivo en reducir las emisiones de carbono y en incentivar la inversión en energías renovables en comparación con no implementar ningún mecanismo de \textit{cap and trade}. En ese artículo se estudian 3 escenarios aplicados a una empresa monopolística. Los escenarios son: Sin mecanismo de cap-and-trade (NM), con \textit{cap and trade} de regla Grandfathering (GM) y mecanismo de \textit{cap and trade} con regla Benchmarking (BM). El estudio concluye que al implementar mecanismos siempre existirá una mejora en el área de inversión en energías no convencionales o en la reducción en emisiones de carbono. Pero también se concluye que un aumento en la inversión en energías renovables no reduce necesariamente las emisiones de carbono totales, debido a que esta inversión puede ser resultado o respuesta únicamente del aumento en generación y no un reemplazo de generación con un nuevo tipo de origen. Se demuestra que la implementación de BM incentiva más este tipo de inversión mientras que con GM se produce menos emisiones de carbono. De esto es importante destacar que el tipo de mecanismo de \textit{cap and trade} afecta en gran medida el resultado, entonces se debe buscar realizar un modelo que finalmente logre las promesas de contaminación propuestas y no implementar cualquier mecanismo. 
\vspace{2.5mm}

\citeB{wang_impact_2021} se focaliza en el efecto que las implementaciones que los mecanismos \textit{Benchmarking} y \textit{Grandfathering} tienen en las cadenas de suministro de las empresas. Se evalúan como los modelos de financiamiento Bank Credit Financing (BCF) y Trade Credit Financing (TCF), en conjunto con BM y GM, afectan las cadenas de suministro de las empresas y cómo estas se comportan. Se observa el comportamiento de un productor y su proveedor. Finalmente, se concluye que sin importar el mecanismo de \textit{cap and trade}, los productores o fabricantes siempre prefieren BCF mientras que los proveedores se inclinan por TCF. También que el tipo de mecanismo que le convenga al productor dependerá principalmente de sus emisiones históricas.
\vspace{2.5mm}

De estos estudios se entiende la importancia en la implementación del tipo de modelo y de cómo el presupuesto de carbono total puede afectar al mercado. A pesar de que esto, el mismo creador del modelo de \textit{cap and trade}, Thomas Crocker, mencionó en el año 2009 ``Soy escéptico de que un modelo de cap-and-trade sea la forma más eficaz de regular el carbono'' \citeB{yarow_inventor_2009}, debido a que él concluía que era difícil predecir cómo este modelo afectará al mercado de generación en el futuro y si no se tiene un ente regulador eficaz, que fiscalice las emisiones continuamente y establezca las capacidades de forma apropiada, un modelo únicamente con impuesto a la emisión parece ser más eficaz. Por lo que un modelo que se pueda regular solo, en el cual exista incentivo en la disminución de emisión, donde también las empresas generadoras tengan más de una oportunidad para cambiar sus inversiones en capacidad y permisos y que este cambio signifique un beneficio para las empresas, sería un camino apropiado para un modelo de \textit{cap and trade} efectivo. \citeB{amigo_two_2021} propone exactamente esto.


\section{Problemas de equilibrio en capacidad} \label{c21}

Para lograr interpretar la toma de decisiones de los agentes en cualquier sistema es necesario modelar los mecanismos en los cuales están involucrados los diferentes actores. Los modelos de optimización bien diseñados logran orientar la toma de decisiones en base a variables que pueden describir la estrategia de un agente. Estas variables son interpretación de qué decisiones deben tomar los agentes involucrados mientras estos tengan el poder de cambiarlos. 
\vspace{2.5mm}

Un modelo bien estructurado, con restricciones basadas en la teoría económica puede significar un cambio operacional y/o táctico para una empresa. Como también un modelo puede mostrar el comportamiento esperado de un mercado ante variaciones de las estrategias.
\vspace{2.5mm}

Los modelos de equilibrio son aquellos que modelan un problema donde los tomadores de decisión (representados por conjuntos de variables y funciones objetivos en el modelo) cumplen con optimalidad individual y restricciones de equilibrio  asociadas a sus decisiones que relaciones las acciones de todos los agentes involucrados. El concepto de solución del problema está representando por un estado estratégico donde no es posible hacer una acción beneficiosa y todos los recursos han sido asignados a su mejor alternativa sin desperdiciar nada. 
\vspace{2.5mm}

El trabajo \citeB{hu_2_2014}  explica que usando un grupo de ecuaciones matemáticas, se puede explicar el equilibrio que existe entre la oferta y demanda en varios mercados. En ese artículo se explica que el modelo incluye variables endógenas y exógenas para representar el sistema, los cambios en estas últimas logran en las decisiones endógenas al sistema.
\vspace{2.5mm}

Esta memoria concentra el estudio en modelos de equilibrio en capacidad relacionados al mercado de generación eléctrica. Donde existe, por una parte, un equilibrio entre la demanda de energía de los consumidores y la oferta de energía de los productores, y por otra parte, un equilibrio entre la capacidad instalada de los productores y los permisos para emitir contaminantes. Siendo este problema formulado como la minimización anual de inversión en las plantas generadoras de energía.
\vspace{2.5mm}

Los modelos de equilibrio en capacidad han sido históricamente estudiados, como en \citeB{ehrenmann_generation_2011}, \citeB{daertrycke_risk_2017}, \citeB{gabriel_complementarity_2012}, \citeB{ralph_risk_2015}, entre otros.
\vspace{2.5mm}

A continuación, en base al trabajo \citeB{daertrycke_risk_2017}, se explica un problema de equilibrio en capacidad con riesgo asociado de dos etapas y uno de riesgo neutral.

\subsection{Problema de equilibrio en capacidad con riesgo asociado (\textit{risky design equilibrium problem}) y modelos de dos etapas}

En este tipo de problemas existe una variable o vector de variables de diseño denotada como  $x \in \mathbb{R}^{n}$, la cual describe la inversión en activos riesgosos tales como plantas de producción de energía que llevan consigo poderes de producción determinados y proporcionales al grado de inversión en mercados con incertidumbre de demanda.
\vspace{2.5mm}

Es importante entender que estos son problemas de optimización estocásticos. Es decir, existe aleatoriedad respecto a los parámetros de entrada. En este caso, la aleatoriedad se ve reflejada en los escenarios estocásticos posibles con $w \in K$  perteneciente a cada escenario $\{1,...,K\}$ de costos. Entonces, existe una entrada de costos y parámetros $C^{w}$ que corresponden a cada escenario $w$. 

La probabilidad de ocurrencia de cada escenario depende de una distribución de probabilidad $\Theta$, la cuál debe ser definida en el problema estudiado y su implicancia, se ve generalmente, en la notación $E_{\Theta}[Z]$. Esta corresponde a la suma ponderada con una probabilidad de ocurrencia para $z_{w} \in Z$ . En otras palabras, $E_{\Theta}[Z]$ es una suma ponderada del valor de una variable en todos los escenarios multiplicados por la probabilidad de ocurrencia de cada escenario. Para ver un ejemplo de lo anterior, revisar Anexo \ref{ej:sumapond}.

En adición a esto, generalmente, estos modelos tienen dos etapas de decisión $i={0,1}$ para los agentes de producción. Donde en la primera etapa $i=0$ el productor (monopolio) o grupo de productores (mercado competitivo) decide su inversión en capacidad x. Luego, en la segunda etapa $i=1$, se decide la producción Y (recordar que si existe más de un escenario, generalmente escenarios que cambian la demanda o costos, la variable de producción también dependerá del escenario w, teniendo $Y_{w}$ para cada uno).
\vspace{2.5mm}

Como explica \citeB{daertrycke_risk_2017}, estos problemas son, en su forma general, descritos como el siguiente. Donde existe una función $f$ convexa y continua, $g$ convexa y continua donde para cada $x$ :

$$\min f(x,y)\quad\text{sujeta a}\quad g(x,y)\le 0$$

Esto es, se busca minimizar los costos asociados al problema y $g(x,y)$ puede ser una o un grupo de restricciones tecnológicas o de equilibrio del sistema.
\vspace{2.5mm}

Ahora, para contrarrestar los costos asociados a escenarios con incertidumbre, por ejemplo, sub o sobre inversión dependiendo de cada escenario, cada agente $i$ se dota de una medida de riesgo $r_{i}$ que mide su costo de incertidumbre $\Xi_{i}(x_{i},x_{-i})$ como $r_{i}=(\Xi_{i}(x_{i},x_{-i}))$. Esta configuración logra que el agente $i$ tenga que tomar otra decisión de inversión, el agente debe decidir qué productos financieros  $W_i$ perteneciente a $W$ elegir para compensar el riesgo de incertidumbre. Estos productos tienen un precio asociado $P^{\tau}[W]$. Este precio es determinado por la condición de que todos los productos financieros deben balancearse logrando que:

$$\sum_{i=1}^{N}W_{i} = 0$$

Para cada una de las dos etapas de este sistema existen importantes consideraciones.
\vspace{2.5mm}

En la primera etapa, la empresa generadora toma una decisión de inversión en capacidad $x$. 
\vspace{2.5mm}

En la segunda etapa, se sabe la demanda del periodo y la empresa generadora produce de forma costo eficiente. La producción es $Y_{w}$ para el escenario $w$ y se vende la energía a un precio $P_{w}\geq 0$ y costo $C_{w}(Y_{w})$. En cada escenario $w$, la empresa buscará maximizar las utilidades o minimizar los costos de operación y, al mismo tiempo, el demandante buscará maximizar sus utilidades de demanda $Q_{w}$ al comprar la energía al precio $P_{w}$.

\subsection{Caso Riesgo Neutral}

Este es el caso donde todos los agentes involucrados en el problema, determinan o consideran a sus costos estocásticos del problema con la probabilidad $\Theta$ anteriormente explicada de cada escenario $w$. Esto significa que aquellos costos asociados a la producción respetan esta probabilidad pero los costos de inversión en capacidad no son la parte estocástica por lo que no son afectados directamente por esta. En el caso neutro al riesgo existe solamente el riesgo en inversión de capacidad pero no un riesgo de compra de activos financieros ya que los agentes, en este caso, son neutrales al riesgo de intercambio o compra de activos financieros. Este es el caso más estudiado en la literatura en comparación a los de mercados completos e incompletos.
\vspace{2.5mm}

Como se mencionó anteriormente, este estudio explora la aplicación de estos modelos en el mercado de empresas generadoras de electricidad, donde existen dos tipos de agentes, empresas generadoras únicamente de electricidad y demandantes de la energía producida. Este es un caso de mercado competitivo por lo que las empresas no actúan estratégicamente para influenciar los precios. Para todos estos casos en particular, se estudiará la situación donde existen 2 agentes en total, 1 generador de electricidad y 1 demandante. 
\vspace{2.5mm}

Bajo estas circunstancias, se tiene que este es un problema competitivo de equilibrio en capacidad con riesgo neutral si es que existe un grupo de variables $(x_{1}:, Y, x_{2}, Q)$ que resuelva lo siguiente: 

$$ \text{min } I_{1}(x_{1})+ I_{2}(x_{2})+E_{\Theta}[C_{w}(Y_{w}-U_{w}(Q_{w})] (1)$$
$$s.a$$
$$ x_{1} \epsilon X_{1} ,x_{2} \epsilon X_{2}$$
$$A_{1w}x_{1}+B_{1w}Y_{w}+b_{1w} \le 0 $$
$$A_{2w}x_{2}+B_{2w}Y_{w}+b_{2w} \le 0 $$
$$Q_{w}\le e^{\tau}Y_{w}\text{  }\text{ para todo w}$$



\section{MCP y Programación modelos de equilibrio}

\subsection{Resolviendo Problemas de equilibrio mediante MCP}\label{descripcionkkt}

La mayoría de los estudios sobre modelos de equilibrio, ya mencionados anteriormente, explican que la mejor forma para resolver los problemas de este tipo, en especial aquellos con gran cantidad de variables o aquellos no lineales, es transformarlos en problemas de comlementaridad mixta o \textit{Mixed Complementarity Problems} (MCP) en inglés.
\vspace{2.5mm}

\citeB{gabriel_complementarity_2012} explican que para realizar esto en necesario primero estructurar el problema de equilibrio (o cualquiera que cumpla el siguiente estilo) de la siguiente forma:

\begin{align}
    \min_{x} & \quad f(x) \label{foej1}\\ 
    \textrm{s.a.} \nonumber\\
    g_{i}(x) \leq 0 ,\quad & i=1,2,...,n  &(\eta_{i}) \label{resej1} 
\end{align}

Con $f(x)$ función objetivo,  $g_{i}(x):\mathbb{R}^n \rightarrow \mathbb{R}$ restricciones y $\eta_i$ los multiplicadores de Lagrange asociados. 
\vspace{2.5mm}

Puede suceder que las restricciones son con sentido inverso y para eso solo basta con multiplicar la restricción \ref{resej1} por $-1$ y así obtener el sistema de ecuaciones deseado. Si \ref{foej1} es una maximización, basta con multiplicarle $-1$ a la función para convertirla en minimización. Para restricciones con igualdad revisar su metodología en \citeB{gabriel_complementarity_2012}.
\vspace{2.5mm}

Luego, se debe calcular el Lagrangeano de este problema con las variables duales de cada restricción ($\eta_{i}$ en \ref{resej1}) de la siguiente forma:

\begin{align}
    \mathcal{L}=f(x) +  \sum_{i=1}^{n}\eta_{i}g_{i}(x)\label{lagraneanoex}
\end{align}

Finalmente, se puede encontrar un valor óptimo para $x$, donde se cumplan las siguientes condiciones de \textit{Karush-Kuhn-Tucker} (KKT). La primera condición \ref{condicion1}, restringe que el gradiente respecto a la variable $x$ del Lagrangeano en \ref{lagraneanoex} debe ser 0 (estacionariedad). La segunda y tercera condición \ref{condicion2}, \ref{condicion3}, establecen la no negatividad de las restricciones (factibilidad). La última condición \ref{condicion4}, establece la complementariedad entre la restricción y su variable dual respectiva\footnote{\ref{condicion2},\ref{condicion3},\ref{condicion4} pueden ser resumidas en una sola condición como: $0\leq\eta_{i}\perp g_{i}(x)\leq 0$}. 

\begin{align}
    0 = \nabla f(x) + \sum_{i=1}^{n} \eta_{i}\nabla g_{i}(x) \label{condicion1}\\
    -g_{i}(x) \geq 0 &, & i=1,2,...,n  \label{condicion2}\\
    \eta_{i} \geq 0 &, & i=1,2,...,n \label{condicion3}\\
    -g_{i}(x)\cdot \eta_{i} = 0 &, & i=1,2,...,n \label{condicion4}
\end{align}

\subsection{Programando modelos de equilibrio}\label{explisolvers}

A pesar de que existen diversos lenguajes de programación que soportan optimización de modelos y muchos de estos contienen librerías especializadas para encontrar estas soluciones, la utilización de programas que soporten problemas NLP (problemas no lineales) es más conveniente de manejar. Esta sub-sección explica el método de cómputo de modelos de equilibrio NLP en \textit{pyscipopt} (librería soportada por Python), GAMS y MCP en GAMS.

\subsubsection{\textit{pyscipopt}}

SCIP es una librería de código abierto para distintos programas. En su página web oficial, se solicita citar los siguientes 2 artículos \citeB{perron_constraint_2008}
y \citeB{achterberg_scip_2009} al utilizar el \textit{solver} para un trabajo. . 
\vspace{2.5mm}

\textit{Pyscipopt} es la interfaz desde python a la \textit{SCIP Optimization Suite}. La ventaja de este \textit{solver}, en adición a entregar resultados certeros de modelos no lineales, es que presenta la posibilidad de incluir sumatorias propias para definir las restricciones con mayor facilidad y su notación simplifica la posibilidad de incluir variables y de utilizar las funciones usuales de Python sin el cuidado de afectar la sintaxis del solver.
\vspace{2.5mm}

Para utilizar este programa es necesario instalar ciertas interfaces con anterioridad. La forma más rápida y eficiente de poder utilizar esta librería es por medio de Google Colab. Una vez en un archivo Colab, es necesario agregar las siguientes líneas de código:


\begin{lstlisting}[language=Python]
!pip install -q condacolab
import condacolab            
condacolab.install()
!conda install --channel conda-forge pyscipopt
\end{lstlisting}
 
Luego, la codificación del problema a resolver debe seguir la siguiente forma y condiciones:

\begin{enumerate}
    \item Comenzar el modelo: se debe escribir el comienzo del modelo o apertura de su diseño con el siguiente comando,
   \begin{lstlisting}[language=Python]
   modelo=Model()
   \end{lstlisting}
   \item Definir las variables: es necesario definir todas las variables con la siguiente nomenclatura,
   \begin{lstlisting}[language=Python]
   x=modelo.addVar('x', vtype='C')  # C variables continuas 
   y=modelo.addVar('y', vtype='d') # d  variables discretas 
   \end{lstlisting}
   \item Generar la función objetivo: se debe definir la función objetivo de forma directa o a partir de otra variable inventada, acá también se define si el problema es de minimización o maximización,
   \begin{lstlisting}[language=Python]
  modelo.setObjective(x+y, "minimize")
  \end{lstlisting}
  \item Agregar las restricciones: hay que agregar las restricciones del modelo, la naturaleza de las variables se pueden agregar en la misma definición de las variables o como restricción,
  \begin{lstlisting}[language=Python]
  modelo.addCons(x>=y)
  modelo.addCons(x>=0)
  modelo.addCons(y>=0)
  \end{lstlisting}
  
  \item  Resolver el problema: se agrega el comando para que el \textit{solver} optimice y así encuentre la mejor solución posible,
  \begin{lstlisting}[language=Python]
  modelo.optimize()
  \end{lstlisting}
  \item  Llamar a las soluciones encontradas: Este paso no es necesario para resolver el problema pero se necesita si se quieren encontrar las soluciones del problema,
  \begin{lstlisting}[language=Python]
  modelo.getVal(x)
  modelo.getObjVal()
  \end{lstlisting}
\end{enumerate}

\subsubsection{GAMS}

Según lo mencionado en la página de \textit{GAMS Studio}\footnote{\url{https://www.gams.com/34/docs/UG_studio_tutorial.html}}, este es un programa que permite la programación e interfaz gráfica para ejecutar GAMS. GAMS, por su lado, es un lenguaje de programación especializado en resolver modelos de optimización. Este sistema es específico para realizar modelamiento de múltiples tipos de modelos de optimización y con múltiples herramientas para resolverlos. Una de las ventajas de utilizar \emph{GAMS Studio} es el detalle de se entrega luego de encontrada (o no) la solución a un problema de modelamiento. Este detalle demuestra los valores de cada variables en detalle junto con sus intervalos de error posibles y entrega información detallada del computador en el cual se está trabajando. En caso de que el sistema no logre encontrar solución, este entregará esa información junto con posibles problemas del modelo. 
\vspace{2.5mm}

Para utilizar este programa es importante entender el problema de optimización escrito, considerando todas las restricciones del problema y definiendo todas las variables y parámetros adecuadamente. Para que no existan errores, se deben utilizar los separadores como “;” y definir cuales corresponden a variables, escenarios, parámetros, ecuaciones, restricciones y nombrar el modelo. Es importante que al definir las restricciones “=g=” significa “mayor que” , “=l=” significa “menor que” y “=e=” significa “igual que”.

Luego, la codificación del problema a resolver debe seguir la siguiente forma y condiciones:

\begin{enumerate}
   \item Definir las variables: es necesario definir las variables dependiendo de su naturaleza. con la siguiente nomenclatura,
   \begin{lstlisting}
  Positive Variable  x; % se define la variable positiva x
  Binary Variable y; % se define la variable binaria y
  Variable z; % se define la variable real z
   \end{lstlisting}
   \item Se mencionan las ecuaciones del modelo: se deben mencionar todas las ecuaciones del modelo, 
   \begin{lstlisting}
 Equations
 fo funcion objetivo
 rest restriccion;
  \end{lstlisting}
  \item Definir las ecuaciones: después de mencionarlas, hay que definir las ecuaciones del modelo,
  \begin{lstlisting}
  fo.. z=e=5x+2;
  rest.. x=g=0;
  \end{lstlisting}
  
  \item  Definir el modelo y resolver: se define el modelo con las ecuaciones y variables antes definidas, se llama al \textit{solver} con el cuál resolverlo y se define si se debe minimizar o maximizar la función objetivo. 
  \begin{lstlisting}
  Model modelo
  /fo,rest/;
  solve modelo using minlp minimizing z;
  \end{lstlisting}
\end{enumerate}


\subsubsection{GAMS con MCP}

Para programar un problema MCP en GAMS se deben seguir las nomenclaturas mencionadas en la subsección anterior. Con excepción del último punto de definición del modelo y como llamar al solver. 

Primero, es necesario entender que para este estilo de resolución de MCP en \textit{GAMS} es necesario tener el problema listo como un MCP. Para esto, una de las opciones es realizarlo mediante las condiciones de KKT explicadas en \ref{descripcionkkt}. Una vez encontradas las condiciones, se deben utilizar cada una de las derivadas de primer orden del Lagraneano en \ref{condicion1} para despejar las variables duales del problema. Una vez realizado esto, y respetando que se cumple \ref{condicion3}, se puede llamar al \textit{solver} para resolver el problema. 

Por ejemplo, se asume que la \ref{condicion1} de un problema de optimización es el siguiente (con $x$ variable primal y $\eta$ dual):

$$0=x-\eta \leftrightarrow \eta=x$$

Luego, se define la ecuación $kkt$ que corresponde a la dual $\eta$:

\begin{lstlisting}
  kkt.. x=g=0;
 \end{lstlisting}

Finalmente, asumiendo que no hay restricciones adicionales y que la variable x está definida previamente como una variable positiva, se define el modelo y se llama al \textit{solver} \textit{PATH} (para MCPs) de la siguiente forma:

\begin{lstlisting}
  options mcp=path;
  model Modelo2
  /kkt.x/
  solve Modelo2 using mcp;
\end{lstlisting}

Es importante entender que la sección \textbf{/kkt.x/} es la que define la condición de complementariedad \ref{condicion4} del MCP. Siendo \textbf{/kkt.x/} el equivalente a  $kkt \perp x$. 

\subsection{Resolución ejemplo 3.3.1 DÁertrycke et al. (2017)}

Para aplicar lo explicado en la sección \ref{explisolvers}, se lleva a cabo la resolución del problema 3.3.1, presente en la página 24 de \citeB{daertrycke_risk_2017}. Este es un problema competitivo de equilibrio en capacidad con riesgo neutral, el cuál está relacionado con el tema central de este trabajo ya que modela la minimización de costos de generación de electricidad al suplir la demanda energética que esta abastece. 
\vspace{2.5mm}

\subsubsection{Explicación Problema}
En este ejemplo existe un productor, el cuál puede invertir únicamente en una tecnología de producción energética. La planta tiene un costo de capital anualizado $I=90 \frac{\textup{\euro}}{kW}$ y costo de operación $C=60 \frac{\textup{\euro}}{kW}$. El año se representa, para este caso, con duración en horas $\tau = 8760 horas$. 
\vspace{2.5mm}

$B = 1\frac{\textup{\euro}}{MWh^2}$ es un parámetro constante en todos lo escenarios $w \in W$ de la utilidad de abastecimiento de demanda $Q_w$: $U_w(Q_w) = A_wQ_w-\frac{B}{2}Q_w^2$.Mientras que el parámetro $A_w$ depende del escenario y tiene los siguientes valores para cada escenario $w$:

\begin{table}[H]
\centering
\begin{tabular}{|l|l|l|l|l|l|}
\hline
\textbf{w} & \textbf{1} & \textbf{2} & 3 & 4 & 5 \\ \hline
A {[$\frac{\textup{\euro}}{MWh}$]} & 300 & 350 & 400 & 450 & 500 \\ \hline
\end{tabular}
\caption{Parámetro de costos}
\end{table}

\vspace{2.5mm}

Este es un modelo de dos etapas:
\begin{enumerate}
    \item[1)] Etapa 0 : la planta toma la decisión de inversión en capacidad instalada para producir en etapa 1.
    \item[2)] Etapa 1: se revela la demanda y se produce energía.
\end{enumerate}

Existen tres variables de decisión en el problema:
\vspace{2.5mm}
\begin{enumerate}
    \item $Q_w: \text{demanda energética en escenario w de etapa 1}$
    \item $Y_w: \text{producción energética en escenario w de etapa 1}$
    \item $x: \text{inversión en capacidad en etapa 0}$
\end{enumerate}

Este problema se modela como una minimización del sistema de la siguiente forma:

\begin{equation}
\begin{array}{rrclcl}
    \displaystyle \min_{Q,x,Y} & Ix+\tau E_{\Theta}[CY_w-A_wQ_w+\frac{B}{2}Q_w^2]  \\\textrm{s.a.} \label{eq:331}
\end{array}
\end{equation}
\begin{equation}
\begin{array}{cl}
    Q_w=Y_w \label{res1:331}
\end{array}
\end{equation}
\begin{equation}
\begin{array}{cl}
   0\leq Y_w \leq x\label{res2:331}
\end{array}
\end{equation}

Donde, \ref{res1:331} restringe que la generación debe ser igual a la demanda y \ref{res2:331} acota la variable de generación entre 0 y la capacidad instalada en la etapa 0.

\subsubsection{Resolución con GAMS}

%Se definen los escenarios del problema:
\begin{lstlisting}
Sets
w possible scenarios /1,2,3,4,5/;
\end{lstlisting}

%Se definen las variables:
\begin{lstlisting}
Positive Variables
x initial investment
*for the producer
Y(w)   production decision
*for the consumer
Q(w) purchased electricity quantity
lambda dual variable of the market clearing constraint;

VARIABLE
z OBJ VALUE;   
\end{lstlisting}

%Se definen parámetros
\begin{lstlisting}
Scalars
I annualized capital expenditure in euro per kilowatt /90/
C operating cost in euro per megawatt per hour /60/
tau duration of single segment in hours /8760/

;

Parameters
B quadratic term of the utility
A(w) term of the utility
Theta(w) physical probability
;

B = 1;
*euro/MWh 

*possible values of A
A('1')=300;
A('2')=350;
A('3')=400;
A('4')=450;
A('5')=500;

Theta('1')=0.2;
Theta('2')=0.2;
Theta('3')=0.2;
Theta('4')=0.2;
Theta('5')=0.2;
\end{lstlisting}

%Ecuaciones
\begin{lstlisting}
Equations
eq1 objective function
mcc(w) market clearing constraint
cap(w)  capacity constraint
;


eq1.. z =e= I*1000*x + tau*sum(w, Theta(w)*(C*Y(w)-A(w)*Q(w)+(B/2)*Q(w)**2));
*factor 1000 para poner las unidades en MW


mcc(w).. Q(w)=e=Y(w);
cap(w).. Y(w)=l=x;   
\end{lstlisting}

\begin{lstlisting}
Model ejemploRN
/eq1,mcc,cap/;

solve ejemploRN using minlp minimizing z;    
\end{lstlisting}


\subsection{Resolución con pyscipopt}

\begin{lstlisting}[language=Python]
from pyscipopt import Model, quicksum

def modeloRNcompetitivecapacityequilibrium(cant_esce,a,theta,
B,tau,c,I):
  m=Model()
  
  # Decision variables t, U, x, Y, Q
  x=m.addVar('x', vtype='C') # Capacity x
  Y={} # Production Y
  Q={} # Demand Q
  
  ## Definition of Production and Demand for each state of
  nature Y_omega, Q_omega
  for i in range(cant_esce):
    Y[i]=m.addVar(vtype="I",name="Y(%s)" % (i))
    Q[i]=m.addVar(vtype="I",name="Q(%s)" % (i))
  ## The next variable is set as the original objetive
  function of the model, it is set as a variable because
  this engine cant support non lineal FO
  objvar = m.addVar(name="objvar", vtype= "C", lb=None,
  ub=None)
  m.setObjective(objvar, "minimize")
  
  # Constraints
  ## Production equals demand    
  for i in range(cant_esce):
    m.addCons(Y[i]==Q[i])
  
  ## Production bounded by installed capacity  
  for i in range(cant_esce):
    m.addCons(Y[i]<=x)  
  
  ## Constraint to satisfy the objetive funtion of the
  original model because of limit in the engine   
  m.addCons(objvar>=(I*1000*x+tau*quicksum(theta[i]*(c*Y[i]-
  a[i]*Q[i]+(B/2)*(Q[i]**2))for i in range(cant_esce)) ))
  m.optimize()
  P=[]
  IM=[]
  for i in range(cant_esce):
    P.append(a[i]-B*m.getVal(Q[i]))
    IM.append((tau*(P[i]-c)*m.getVal(Y[i])
    -I*m.getVal(x))/m.getVal(x))
  Qprom=0
  Yprom=0
  for i in range(cant_esce):
    print ("Q(%s)="%(i+1) , m.getVal(Q[i]),"Y(%s)= "%(i+1) ,
    m.getVal(Y[i]),"   P(%s)= "%(i+1), P[i],
    "Investment Margin= ", IM[i] )
    Qprom+=m.getVal(Q[i])
  print ("QPromedio= " , Qprom/cant_esce)
  print ("x= " , m.getVal(x))
  print ("Valor funcion objetivo = ", m.getObjVal())

  print (cant_esce,a,theta,B,tau,c,I)

cant=5
A=[300,350,400,450,500]
Theta=[0.2,0.2,0.2,0.2,0.2]
B=1  #$/MWh
tau= 8760 #duracion por segmento por hora
C=60 #costo de operacion en $/mWhora
I=90 #capital para gasto anualizado en $/MWh

modeloRNcompetitivecapacityequilibrium(cant,A,Theta,B,tau,C,I) 
\end{lstlisting}

\subsubsection{Comparación de resultados}
Finalmente, se comparan los resultados del modelo en ambos \textit{solvers} junto con los resultados presentes en \citeB{daertrycke_risk_2017}.

\begin{table}[H]
\centering
\begin{tabular}{|l|l|l|l|l|l|l|l|l|l|}
\hline
w & Q{[}MWh{]} & Q* & Q** & P{[\textup{\euro}]/MWh} & P* & P** & x{[}MW{]} & x* & x** \\ \hline
1 & 240 & 240 & 240 & 60 & 60 & 60 & 389 & 389,31 & 388,99 \\ \hline
2 & 290 & 290 & 290 & 60 & 60 & 60 &  &  &  \\ \hline
3 & 340 & 340 & 340 & 60 & 60 & 60 &  &  &  \\ \hline
4 & 389 & 389,32 & 388,99 & 60,7 & 60,68 & 61 &  &  &  \\ \hline
5 & 389 & 389,31 & 388,99 & 110,7 & 110,69 & 111 &  &  &  \\ \hline
Promedio & 329,6 & 331,52 & 329,596 & 70,3 & 70,3 & 70,4 &  &  &  \\ \hline
\multicolumn{10}{l}{\footnotesize () valor  DÁertrycke et al. (2017) , (*) valor obtenido en \textit{GAMS}, (**) valor obtenido en \textit{pyscipopt}}\\
\end{tabular}
\caption{Resultados ejemplo}
\label{tabla:ejemplos}
\end{table}

El cuadro \ref{tabla:ejemplos} muestra que el rendimiento de ambos programas entregó resultados similares a aquellos presentes en el trabajo original.



\section{Amigo, Cea-Echenique \& Feijoo, 2021}

Como se menciona en \ref{c22}, los mecanismos de \textit{cap and trade}, por lo menos en teoría, pueden funcionar de forma efectiva cuando se aplica el mecanismo correcto para el problema que se quiere solucionar. El modelo de \textit{cap and trade} de dos etapas propuesto por los autores Pía Amigo, Sebastián Cea-Echeñique y Felipe Feijoo, en su trabajo titulado \textit{A two stage cap-and-trade model with allowance re-trading and capacity investment: The case of the Chilean NDC targets}, busca disminuir las emisiones de $CO_2$ e incentivar el cambio de generación de energía a fuentes más sustentables al aplicar un modelo que reconoce que tanto los productores como el planificador social (subastador o regulador) buscan maximizar sus beneficios.
\vspace{2.5mm}

En él, el modelo propone un sistema de dos etapas involucrando a los generadores de electricidad y al subastador que proporciona los permisos de emisiones de dióxido de carbono ($CO_2 e$). En la primera etapa (periodo $t=0$), el subastador determina los permisos de contaminación que serán vendidos a los productores de electricidad. Estos permisos corresponden a la contaminación máxima que cada generador puede emitir. Por otro lado, en la misma etapa, las empresas generadoras deciden su inversión en capacidad, su generación y la asignación de sus permisos para satisfacer una demanda exógena. Ellos compran los permisos $A_i$ al subastador a un precio $\pi^a$ para poder hacer viable, en términos de permisos de emisión de contaminante, su producción. 
\vspace{2.5mm}

La segunda etapa se distribuye entre el periodo siguiente a la asignación inicial de permisos por parte del subastador hasta el fin del período analizado ($t=1,...,T$). Particularmente en $t=1$ es donde se revela la demanda futura dada por un estado de la naturaleza $w \in W :=\{ 1,..,K\}$ donde su probabilidad de ocurrencia está definida por $P(w)$. Según esta demanda, los productores recalculan sus niveles de generación, determinar nuevamente su capacidad e inversión en permisos faltantes. También tienen la oportunidad de comprar permisos o vender los suyos para adaptar su esperanza de generación en el futuro. Esto les otorga una segunda oportunidad para adaptar sus variables de generación sin ser afectados en gran manera por la incertidumbre de demanda.
\vspace{2.5mm}

\begin{figure}[htp]
    \centering
    \includegraphics[width=15cm]{docs/DocumentoMemoria/images/Tabla amigo.png}
    \caption{Nomenclatura modelo original (Fuente: \protect\citeB{amigo_two_2021})}
    \label{fig:nomenclatura1}
\end{figure}

La Figura \ref{fig:nomenclatura1} define todas las variables del modelo original para así simplificar la lectura las ecuaciones por definir a continuación.
\vspace{2.5mm}

El modelo completo de \textit{cap and trade} propuesto en el artículo principal es dividido en dos modelos separados de equilibrio en capacidad que modelan el problema de los agentes (subastador y productores) donde ellos buscan maximizar sus beneficios.
\vspace{2.5mm}

Por un lado, está el problema de los productores en \ref{fo:prod}. Esta está acotada por restricciones de capacidad en generación en \ref{res:1}, restricción en generación inicial \ref{res:2}, de limites en capacidad \ref{res:3} y venta y compra de permisos \ref{res:4} y \ref{res:5}. 

\begin{align}
\min_{(x_i,Q_i,A_i,P_i,V_i)\in \mathbb{X}} &  f_i \big( \pi^d(0),Q_i(0)\big)+ A_i \pi^{a} + I_i x_i(0) \nonumber \\ 
& + \sum_{\omega} Pr(\omega)   \Bigg[ \sum_{t>0} \frac{1}{(1+R)^t} \Big[ TC_i(t,\omega)\cdot f_i \big( \pi^d(t,\omega),Q_i(t,\omega) \big)  \nonumber \\
 & + TCR_i(t,\omega) \cdot I_i\cdot x_i(t,\omega) \Big] + \pi^v(\omega)\cdot \big(P_i(\omega)-V_i(\omega)\big) \Bigg]  \label{fo:prod}\\
     \textrm{s.t \ } \nonumber
\end{align}
\begin{align}
    \Big(CF_i \cdot\tau\Big)  \Bigg[\bar{Q}_i + \sum_{t^{\prime}<\bar{t}} x_i(t^\prime,\omega) + x_i(0)+ \bar{Q}_i(t) \Bigg] - Q_i(t,\omega) & \geq 0  & \forall  \quad i,\omega, t  > 0 & \quad (\alpha_{i,\omega,t})\label{res:1}\\
    \Big(CF_i\cdot\tau \Big)\bar{Q_i}-Q_{i}(0) & \geq 0  & \forall  \quad i & \quad (\kappa_i) \label{res:2}\\
     RP_i - \bar{Q}_i  - x_i(0) - \sum_{t > 0} x_i(t,\omega) & \geq 0 &  \forall \quad i,\omega &   \quad (\psi_{i,\omega}) \label{res:3}\\
 A_{i} -V_i(\omega) & \geq  0  & \forall  \quad \omega & \quad (\beta_{i,\omega}) \label{res:4}\\
 A_{i} + (P_i(\omega) - V_i(\omega))-\sum_{t>0}Q_i(t, \omega)\cdot \varepsilon_{i}-Q_i(0)\varepsilon_{i} & \geq  0  &\forall \quad \omega & \quad (\gamma_{i,\omega})\label{res:5}\\
 Q_i(0) & \geq  0 & \forall \quad i & \quad (\lambda_i) \label{res:q0}\\ 
 Q_i(t, \omega) & \geq  0   & \forall \quad \omega, t >0 & \quad (\delta_{i,\omega,t})\label{res:qt}\\
  x_i(0) & \geq  0 & \forall  \quad i & \quad (\xi_i)  \label{res:capi0}\\ 
  x_i(t, \omega) & \geq  0   & \forall  \quad \omega, t >0 & \quad (\varphi_{i,\omega,t})\label{res:capt}
  \end{align}

Es de importancia mencionar que las variables presentes al final de cada restricción (por ejemplo $\varphi_{i,\omega,t}$ en \ref{res:capt}) corresponden a las variables duales de cada una respectivamente.
\vspace{2.5mm}

Por otro lado, el problema del subastador está formulado de la siguiente forma en \ref{eq:sub}. En la sección siguiente se explica con más detalle este modelo y sus restricciones.

\begin{equation}
\begin{array}{rrclcl}
    \displaystyle \min_{\theta} &-\theta \pi^a + F(\theta) \\\textrm{s.a.} \label{eq:sub}\\
\end{array}
\end{equation}
\begin{equation}
\begin{array}{cl}
    \varphi^-1 (\varepsilon )\sigma + \mu - \theta \geq 0 & (\eta) \label{res:sub1}
\end{array}
\end{equation}
\begin{equation}
\begin{array}{cl}
   \theta \geq 0 & (\varrho)\label{res:sub2}
\end{array}
\end{equation}

Con la finalidad de lograr que ambos modelos se relacionen y puedan ser resueltos en un único sistema de ecuaciones como MCP, las variables de ambos problemas deben resolver las siguientes condiciones de mercado.

\begin{align}
   \sum_{i}A_i = \theta  &\quad (\pi^a)\label{rescom:1}
\end{align}

Donde se cumple que los permisos disponibles en el primer periodo no puede superar a los permisos emitidos por el subastador.

\begin{align}
    \sum_{i}P_{i,\omega} = \sum_{i}V_{i,\omega} \quad& \forall \omega &(\pi^v (w))\label{rescom:2}
\end{align}
    


Esta explica que debe habe equilibrio en el mercado de compra y ventas de permisos.

\begin{align}
  \sum_{i}Q_i(0) = D(0) \quad (\pi^d (0))\label{rescom:3}  
\end{align}


La demanda energética debe ser abastecida en la primera etapa.

\begin{align}
    \sum_{i}Q_i(t,\omega) = D(t,\omega) \quad& \forall  \omega,t & (\pi^d (\omega,t))\label{rescom:4}
\end{align}

Finalmente \ref{rescom:4} restringe que la demanda sea igual a la producción en todos los periodos y escenarios de la segunda etapa.
\vspace{2.5mm}

Evaluando de forma general este modelo, el sistema le permite a los generadores modificar sus sistemas en la segunda etapa para así poder obtener más utilidad al poder ajustar sus emisiones.
\vspace{2.5mm}

Esta oportunidad de reajuste le otorga interesantes oportunidades tanto a las empresas generadoras como al gobierno que quiere disminuir las emisiones. Por un lado, esta instancia les permite a las empresas mejorar sus ganancias acorde a las proyecciones de demanda. Por otro lado, el gobierno o ente regulador tendrá la opción de forzar ventas de permisos a las empresas de carbón para así incentivarles el cambio a opciones más renovables. 
\vspace{2.5mm}

El trabajo finalmente concluye que con el actual impuesto al carbono utilizado en Chile, no se llegaría a los objetivos de eliminar el carbón como fuente de generación para 2030 ni para ser carbono neutral en el año 2050. Pero proponen que definir un presupuesto de carbono (\textit{CAP}) entre 500-600 $MtCO_2 e$ o menos eliminaría al carbón en 15 años. También que con un \textit{CAP} menos severo (mayor $MtCO_2 e$) aumentaría la producción por energía con fuentes no convencionales(sustentables) a un 60-70\% del total de empresas generadoras pero no eliminaría totalmente al carbón en el tiempo prometido. Y como tercer descubrimiento, se concluye que si se añaden restricciones adicionales a las empresas generadoras (como imponer que estas deben vender sus permisos al llegar al año 2035), los objetivos de eliminar el carbón puede ser obtenible.
\vspace{2.5mm}

En el análisis anterior se reconoce la importancia del subastador en este modelo y en el futuro de la generación eléctrica si se planea utilizar este sistema de \textit{cap and trade}. La cantidad de permisos emitidos según el \textit{CAP} deben estar de acuerdo a un criterio de optimalidad. Es decir, lo más cercanos a la realidad en cuanto al presupuesto determinado y la demanda futura. Es por esto que se decide profundizar en el problema del subastador para hacer más realista su rol en el sistema.
\vspace{2.5mm}

\section{Transformación del modelo original de Amigo et al. (2021) a un MCP}

Una de las formas de resolver el modelo original de \citeB{amigo_two_2021}, gracias a su forma, es transformarlo en un MCP. Esto es posible por medio del teorema de Karush-Kuhn-Tucker \ref{descripcionkkt} y sus condiciones resultantes. 
\vspace{2.5mm}

Por lo tanto, para replicar el modelo original, primero se realiza esta transformación. Para esto, es necesario entender que el modelo creado consiste en dos modelos que optimizan las utilidades de dos grupos de agentes distintos. Por un lado, se optimizan las utilidades de los productores de electricidad \ref{fo:prod} y, por otro lado, está el modelo que optimiza al subastador o ente regulador de permisos de contaminación \ref{eq:sub}. Para resolver el modelo como un solo problema MCP es necesario juntar ambos en un único problema de tipo MCP. Para lograr esto, además de encontrar las ecuaciones de KKT de cada uno de los grupos de agentes, es necesario incorporar las 4 condiciones de mercado \ref{rescom:1}, \ref{rescom:2},\ref{res:3},\ref{rescom:4} que relacionan ambos grupos de agentes.

\subsection{MCP problema del productor}

Primero, es necesario aplicar el teorema de KKT en  el problema del productor \ref{fo:prod} para transformarlo en un MCP.
\vspace{2.5mm}

El lagrangeano del problema del productor queda: 

\footnotesize{
\begin{align}
&\mathcal{L}_i(x_i,Q_i,A_i,P_i,V_i) = f_i \big( \pi^d(0),Q_i(0)\big)+ A_i \pi^{a} + I_i x_i(0)  +& \nonumber \\ 
&\sum_{\omega} Pr(\omega)\Bigg[ \sum_{t>0} \frac{1}{(1+R)^t} \Big[ TC_i(t)\cdot f_i \big( \pi^d(t,\omega),Q_i(t,\omega) \big) + TCR_i(t) \cdot I_i\cdot x_i(t,\omega) \Big] + \pi^v(\omega)\cdot \big(P_i(\omega)-V_i(\omega)\big) \Bigg]   + &\nonumber \\
&\kappa_{i}\Big[Q_i(0) -  \big(CF_i\cdot\tau \big)\bar{Q}_i \Big] +\sum_{\omega,t>0} \alpha_{i,\omega,t}\Bigg[Q_i(t,\omega) - \big(CF_i \cdot\tau\big) \big(\bar{Q}_i + \sum_{t^{\prime} \leq t } x_i(t,\omega) + x_i(0) \big)\Bigg] + & \nonumber \\ &\sum_{\omega}\beta_{i,\omega}\Big[V_i(\omega)-A_i \Big] + \sum_{\omega}\gamma_{i,\omega} \Big[-A_{i} - P_{i}(\omega) + V_i(\omega) +\sum_{t>0} Q_i(t,\omega) \varepsilon_{i} + Q_i(0)\varepsilon_{i}\Big] - \sum_{\omega, t>0}\delta_{i,\omega,t} Q_i(t,\omega) + & \nonumber \\ &\sum_{\omega}\psi_{i,\omega} \Big[  \bar{Q}_i+ x_i(0) + \sum_{t > 0} x_i(t,\omega) - RP_i \Big] - \lambda_{i}\Big[Q_{i}(0)\Big] - \sum_{\omega, t>0}\varphi_{i,\omega,t} x_i(t,\omega) - \xi_i x_i(0) & \label{eq:lagrange}
\end{align}}

Realizando las derivadas de primer orden se obtienen:

\subsubsection{Derivada parcial respecto a $x_i(0)$}
\footnotesize{
\begin{align}
    \frac{\partial \mathcal{L} }{\partial x_i(0)} = 0 = I_i  + \sum_{\omega}\psi_{i,\omega} -\sum_{\omega, t>0} \alpha_{i,\omega,t}(CF_i\cdot \tau) -\xi_i=0  \qquad \forall \  i \\
    \Leftrightarrow I_i  + \sum_{\omega}\psi_{i,\omega} -\sum_{\omega, t>0} \alpha_{i,\omega,t}(CF_i\cdot \tau) = \xi_i  \qquad \forall \  i 
\end{align}
}

Donde $\xi_i$ es la variable dual en \ref{res:capi0}. Por lo que esta kkt debe ser complementaria a $x_i(0)$, obteniendo la siguiente complementariedad:

\footnotesize{
\begin{align}
    x_i(0)\geq 0 \qquad \forall \  i \\
    I_i  + \sum_{\omega}\psi_{i,\omega} -\sum_{\omega, t>0} \alpha_{i,\omega,t}(CF_i\cdot \tau) \geq 0  \qquad \forall \  i\\
    x_i(0)\cdot(I_i  + \sum_{\omega}\psi_{i,\omega} -\sum_{\omega, t>0} \alpha_{i,\omega,t}(CF_i\cdot \tau))=0 \qquad \forall \  i
\end{align}
}
En el código GAMS , se escribió la kkt de esta complementariedad de la siguiente forma en la línea 654:
\begin{verbatim}
kkt_x_first_producer(i).. Inv(i)*(1+percent(i)) + sum(w, psi(i,w))- CF(i)*t_year
*sum(w, sum( tau2,alpha(i,w,tau2))) - xi(i)=g=0;
\end{verbatim}
Siendo xi(i) la variable dual $\xi_i$ . Donde esta kkt se complementa con la variable respectiva $x_i(0)$ en la línea 724 del código.
\begin{verbatim}
kkt_x_first_producer.x_first
\end{verbatim}

\subsubsection{Derivada parcial respecto a $x_i(t,\omega)$}
\footnotesize{
\begin{align}
    \frac{\partial \mathcal{L} }{\partial x_i(t,\omega)}= 0 = Pr(\omega) \Bigg[\frac{1}{(1+R)^t}TCR_i(t) \cdot I_i \Bigg] - \sum_{t> t\prime}\alpha_{i,\omega,t} ( CF_i \cdot \tau)+ \psi_{i,\omega}-\varphi_{i,\omega,t} \qquad  \forall \  i, \omega, t> 0\\
    \leftrightarrow Pr(\omega) \Bigg[\frac{1}{(1+R)^t}TCR_i(t) \cdot I_i \Bigg] - \sum_{t> t\prime}\alpha_{i,\omega,t} ( CF_i \cdot \tau)+ \psi_{i,\omega}=\varphi_{i,\omega,t} \qquad  \forall \  i, \omega, t> 0
\end{align}
}

Donde $\varphi$ es la variable dual en la restricción \ref{res:capt}, por lo que, para resolver como MCP se tiene la siguiente complementariedad:


\footnotesize{
\begin{align}
    Pr(\omega) \Bigg[\frac{1}{(1+R)^t}TCR_i(t) \cdot I_i \Bigg] - \sum_{t> t\prime}\alpha_{i,\omega,t} ( CF_i \cdot \tau)+ \psi_{i,\omega} \geq 0 \qquad  \forall \  i, \omega, t> 0\\
    x_i(t,\omega) \geq 0 \qquad  \forall \  i, \omega, t> 0\\
    (Pr(\omega) \Bigg[\frac{1}{(1+R)^t}TCR_i(t) \cdot I_i \Bigg] - \sum_{t> t\prime}\alpha_{i,\omega,t} ( CF_i \cdot \tau)+ \psi_{i,\omega})*x_i(t,\omega)=0
\end{align}
}

Al igual que en el caso anterior, el código mantiene la variable dual, se puede ver en la línea 657 del código el KKT respectivo:

\begin{verbatim}
kkt_x_second_producer(i,tau2,w).. ((1/(1+R))**(ord(tau2)))*Prob(w)*Inv(i)*
(1+percent(i))*TCR(i,tau2)- CF(i)*t_year*sum(tau3$(tauAlpha_i(i,tau2,
tau3)),alpha(i,w,tau3)) + psi(i,w) -varphi(i,w,tau2)  =g=0;
\end{verbatim}

Con varphi(i,w,tau2) como la variable dual mencionada $\varphi_{i,\omega,t}$. 


\subsubsection{Derivada parcial respecto a $Q_i(0)$}
\footnotesize{
\begin{align}
    \frac{\partial \mathcal{L} }{\partial Q_i(0)}= 0 =  \big(a_{i}+b_i Q_{i}(0)\big)-\pi^d(0) + \kappa_i  + \sum_{\omega} \gamma_{i,\omega}\varepsilon_i-\lambda_i \qquad \forall \  i  \\
     \leftrightarrow \big(a_{i}+b_i Q_{i}(0)\big)-\pi^d(0) + \kappa_i  + \sum_{\omega} \gamma_{i,\omega}\varepsilon_i = \lambda_i \qquad \forall \  i
\end{align}
}
$\lambda_i$ es la variable dual de la restricción \ref{res:q0}, por lo que, para resolver como MCP se tiene lo siguiente:
\footnotesize{
\begin{align}
    \big(a_{i}+b_i Q_{i}(0)\big)-\pi^d(0) + \kappa_i  + \sum_{\omega} \gamma_{i,\omega}\varepsilon_i \geq 0 \qquad \forall \  i  \\
    Q_i(0) \geq 0 \qquad \forall \  i  \\
    Q_i(0)*( \big(a_{i}+b_i Q_{i}(0)\big)-\pi^d(0) + \kappa_i  + \sum_{\omega} \gamma_{i,\omega}\varepsilon_i)=0 
\end{align}
}
Como en los anteriores, en el código no se despeja la variable dual $\lambda_i$, se mantiene y se programa el MCP respectivo con la variable $Q_i(0)$. Se puede ver en la línea 660 del código de la siguiente forma:

\begin{verbatim}
kkt_q1_producer(i,tau).. (int(i)+C(i)*Q_first(i,tau))-pi_d(tau) + kappa(i,tau) - 
lambda(i,tau)+sum(w,gamma(i,w)*epsilon(i))=g= 0;
\end{verbatim}

Programando la complementariedad en la línea 728:

\begin{verbatim}
kkt_q1_producer.Q_first    
\end{verbatim}


\subsubsection{Derivada parcial respecto a $Q_i(t,\omega)$}
\footnotesize{
\begin{align}
   \frac{\partial \mathcal{L} }{\partial Q_i(t,w)}= 
   0= Pr(\omega)  \frac{1}{(1+R)^t} \bigg( TC_i(t) \big(a_{i}+b_i Q_i(t,\omega)\big ) -\pi^d(t,\omega) \bigg) + \alpha_{i,\omega,\tau} + \gamma_{i,\omega} \varepsilon_{i} -\delta_{i,\omega,t} \qquad  \forall \ i, \omega, t > 0\\
   \leftrightarrow Pr(\omega)  \frac{1}{(1+R)^t} \bigg( TC_i(t) \big(a_{i}+b_i Q_i(t,\omega)\big ) -\pi^d(t,\omega) \bigg) + \alpha_{i,\omega,\tau} + \gamma_{i,\omega} \varepsilon_{i}= \delta_{i,\omega,t} \qquad \forall \ i, \omega, t > 0
\end{align}
}
$\delta_{i,\omega,t}$ es la variable dual de la restricción \ref{res:qt}, al resolverlo como MCP se tiene la siguiente complementariedad:

\footnotesize{
\begin{align}
    Pr(\omega)  \frac{1}{(1+R)^t} \bigg( TC_i(t) \big(a_{i}+b_i Q_i(t,\omega)\big ) -\pi^d(t,\omega) \bigg) + \alpha_{i,\omega,\tau} + \gamma_{i,\omega} \varepsilon_{i} \geq 0 \qquad \forall \ i, \omega, t > 0\\
    Q_i(t,\omega) \geq 0 \qquad  \forall i, \omega, t > 0\\
    (Pr(\omega)  \frac{1}{(1+R)^t} \bigg( TC_i(t) \big(a_{i}+b_i Q_i(t,\omega)\big ) -\pi^d(t,\omega) \bigg) + \alpha_{i,\omega,\tau} + \gamma_{i,\omega} \varepsilon_{i})\cdot  Q_i(t,\omega)= 0 \qquad \forall \ i, \omega, t > 0
\end{align}
}

La línea 662 del código completa la kkt de la siguiente forma: 

\begin{verbatim}
kkt_qtau_producer(i,tau2,w).. 
((1/(1+R))**(ord(tau2)))*Prob(w)*((TC(i,tau2)*(int(i)+C(i)*Q_second(i,tau2,w)))- 
\end{verbatim}
\begin{verbatim}
pi2_d(tau2,w))+alpha(i,w,tau2)+gamma(i,w)*epsilon(i)-delta(i,tau2,w) =g= 0;
\end{verbatim}

Con la complementariedad escrita de la siguiente forma en la línea 729 del código:
\begin{verbatim}
kkt_qtau_producer.Q_second    
\end{verbatim}

\subsubsection{Derivada parcial respecto a $A_i(t)$}
\footnotesize{
\begin{align}
   \frac{\partial \mathcal{L} }{\partial A_i}= 0
   = \pi^{a} - \sum_{\omega}\beta_{i,\omega} - \sum_{\omega}\gamma_{i,\omega}  \qquad \forall \  i \\
\end{align}
}

Para resolverlo como MCP se tiene la siguiente complementariedad:
\footnotesize{
\begin{align}
    \pi^{a} - \sum_{\omega}\beta_{i,\omega} - \sum_{\omega}\gamma_{i,\omega} \geq 0 \qquad \forall  i\\
    A_i \geq 0 \qquad \forall  i\\
    (\pi^{a} - \sum_{\omega}\beta_{i,\omega} - \sum_{\omega}\gamma_{i,\omega}) \cdot A_i = 0 \qquad \forall  i
\end{align}
}

En el código se tiene lo siguiente en la línea 664:
\begin{verbatim}
   kkt_A_producer(i).. pi_a - sum(w, beta(i,w)) - sum(w, gamma(i,w))=g= 0 
\end{verbatim}

Y efectuando la complementariedad en la línea 730:
\begin{verbatim}
    kkt_A_producer.A
\end{verbatim}

\subsubsection{Derivada parcial respecto a $V_i(w)$}
\footnotesize{
\begin{align}
   \frac{\partial \mathcal{L} }{\partial V_i(w)}= 0
   = -Pr(\omega) \pi^v(\omega) + \beta_{i,\omega}  + \gamma_{i,\omega}  \qquad \forall \  i, \omega \\
\end{align}
}

Para resolverlo como MCP se tiene la siguiente complementariedad:
\footnotesize{
\begin{align}
    -Pr(\omega) \pi^v(\omega) + \beta_{i,\omega}  + \gamma_{i,\omega} \geq 0 \qquad \forall \  i, \omega \\
    V_i(w) \geq 0 \qquad \forall  i,\omega \\
    (-Pr(\omega) \pi^v(\omega) + \beta_{i,\omega}  + \gamma_{i,\omega}) \cdot  V_i(w) = 0  \qquad \forall \  i, \omega 
\end{align}
}

Programando la kkt en el código en la línea 667:
\begin{verbatim}
    kkt_V_producer(i,w)..  -Prob(w)*pi_v(i,w)+beta(i,w)+gamma(i,w)=g= 0
\end{verbatim}

Completando la complementariedad en la línea 731:
\begin{verbatim}
    kkt_V_producer.V
\end{verbatim}

\subsubsection{Derivada parcial respecto a $P_i(w)$}
\footnotesize{
\begin{align}
   \frac{\partial \mathcal{L} }{\partial P_i(w)}= 0
   = Pr(\omega) \pi^v(\omega) -\gamma_{i,\omega} \qquad \forall \  i, \omega \\
\end{align}
}

Para resolverlo como MCP se tiene la siguiente complementariedad:
\footnotesize{
\begin{align}
    Pr(\omega) \pi^v(\omega) -\gamma_{i,\omega} \geq 0 \qquad \forall \  i, \omega \\
    P_i(w) \geq 0 \qquad \forall  i,\omega \\
    (Pr(\omega) \pi^v(\omega) -\gamma_{i,\omega}) \cdot  P_i(w) = 0  \qquad \forall \  i, \omega 
\end{align}
}

Programando la kkt en el código en la línea 670:
\begin{verbatim}
    kkt_P_producer(i,j,w)$(ord(j) <> ord(i))..      Prob(w)* pi_v(j,w)-gamma(i,w) =g= 0
\end{verbatim}

Completando la complementariedad en la línea 732:
\begin{verbatim}
    kkt_P_producer.P  
\end{verbatim}

\subsubsection{Completando el problema del productor}
El resto de las complementariedades del problema del productor se producen al complementar las restricciones del modelo original con sus variables duales:

\footnotesize{
\begin{align}
    & 0 \leq \big(CF_i \cdot \tau \big) \Bigg[\bar{Q}_i + \sum_{t\leq t^{\prime}} x_i(t,\omega) + x_i(0) \Bigg] - Q_i(t,\omega)  \perp \alpha_{i,\omega,\tau} \geq 0 \qquad \forall \ i, \omega, t  > 0\\
    & 0 \leq \Big(CF_i\cdot\tau \Big)\bar{Q}_i(0)-Q_{i}(0) \perp \kappa_i \geq 0 \qquad \forall \ i \\
    & 0 \leq  A_{i} - V_i(\omega) \perp \beta_{i,\omega} \geq 0 \qquad \forall  \ \omega \\
    & 0 \leq  A_{i} + P_{i} (\omega) - V_i(\omega) - \sum_{t>0} Q_i(t,\omega) \varepsilon_{i} -Q_i(0)\varepsilon_{i} \perp \gamma_{i,\omega} \geq 0 \qquad \forall \ i, \omega\\
    & 0 \leq  RP_i - \bar{Q}_i - x_i(0) - \sum_{t>0} x_i(t,\omega) \perp \psi_{i,\omega} \geq 0 \qquad \forall \ i,\omega 
\end{align}
}
Las cuales se encuentran programadas entre las líneas 609 y 703 del código:

\begin{verbatim}
    capacity_stage_2_xnext(i,w,tau2).. CF(i)*t_year*(sum(tau3$(tau3_i(i,tau2,tau3)),
    x_next(i,tau3,w))+x_first(i)+Q_barT2(i,tau2) +Q_bar(i))-Q_second(i,tau2,w) =g=0;
\end{verbatim}
\begin{verbatim}
    capacity_stage_1(i,tau)..  CF(i)*t_year*Q_bar(i)- Q_first(i,tau) =g=0;
\end{verbatim}
\begin{verbatim}
    trading_permits(i,w)..  A(i)-V(i,w) =g= 0;
\end{verbatim}
\begin{verbatim}
    total_allowances(i,w)..  A(i) + sum(j$(ord(j) <> ord(i)),P(i,j,w))-V(i,w) - 
    sum(tau2,Q_second(i,tau2,w)*epsilon(i)) - sum(tau,Q_first(i,tau)*epsilon(i))   =g= 0;
\end{verbatim}
\begin{verbatim}
    resource_potential(i,w)..   -Q_bar(i) - x_first(i) - sum(tau3,x_next(i,tau3,w))+
    sum(tau2,Q_barT2(i,tau2)) + RP(i) =g= 0;
\end{verbatim}

Existiendo sus complementariedades entre las líneas 734 y 744:

\begin{verbatim}
    capacity_stage_1.kappa
    capacity_stage_2_xnext.alpha
    resource_potential.psi
    trading_permits.beta
\end{verbatim}

\subsection{MCP problema del subastador}

Debido a que esta memoria se centra en la mejora e innovación de este modelo, este MCP se incluye en el siguiente capítulo \ref{MCP:subastador}. En el cuál, luego de encontrar las restricciones, se proporcionan una reestructuración de este. 

\subsection{Condiciones de mercado}
Para lograr unir los modelos del subastador y productor en un solo MCP es necesario incluir las restricciones de condiciones de mercado \ref{rescom:1}, \ref{rescom:2},\ref{res:3},\ref{rescom:4}. De estas se obtienen las siguientes complementariedades:

\subsubsection{Restricción de permisos disponibles}
\footnotesize{
\begin{align}
 \pi^a \geq 0 \\
 \theta - \sum_{i}A_i = 0  
\end{align}}
En el código se encuentra la restricción y complementariedad en las líneas 708 y 751 respectivamente: 
\begin{verbatim}
    kkt_A_theta..    theta - sum(i,A(i))=e= 0;
    kkt_A_theta.pi_a
\end{verbatim}

\subsubsection{Restricción de equilibrio en el mercado de compra y venta de permisos}
\footnotesize{
\begin{align}
 \pi^v(\omega) \geq 0 \qquad \forall \omega\\
 \sum_{i}P_{i,\omega} - \sum_{i}V_{i,\omega} = 0 \qquad \forall \omega  
\end{align}}
En el código se encuentra la restricción y complementariedad en las líneas 713 y 752 respectivamente: 
\begin{verbatim}
    kkt_P_V(i,w).. V(i,w)=e=sum(j$(ord(j) <> ord(i)),P(j,i,w));
    kkt_P_V.pi_v 
\end{verbatim}

\subsubsection{Abastecimiento de demanda en la primera etapa}
\footnotesize{
\begin{align}
 \pi^d(0) \geq 0 \\
 \sum_{i}Q_i(0) - D(0) = 0   
\end{align}}
En el código se encuentra la restricción y complementariedad en las líneas 710 y 753 respectivamente: 
\begin{verbatim}
    kkt_Q_D_first(tau).. sum(i,Q_first(i,tau))=e=D1(tau);
    kkt_Q_D_first.pi_d
\end{verbatim}

\subsubsection{Abastecimiento de demanda en la segunda etapa}
footnotesize{
\begin{align}
 \pi^d(\omega,t) \geq 0 \qquad \forall \omega,t\\
 \sum_{i}Q_i(t,\omega) - D(t,\omega) = 0 \qquad \forall \omega,t  
\end{align}}
En el código se encuentra la restricción y complementariedad en las líneas 711 y 754 respectivamente: 
\begin{verbatim}
    kkt_Q_D_second(tau2,w)..sum(i,Q_second(i,tau2,w))=e=D2(w,tau2);
    kkt_Q_D_second.pi2_d
\end{verbatim}


\section{Costos de información y su aplicación en problemas de maximización de beneficios}\label{marco:costos}

En el contexto de modelos de optimización en economía, por ejemplo la maximización de utilidades, diversos factores pueden afectar su función objetivo. En varios casos se reconoce la existencia de ``ruidos'' que afectan el rendimiento de una inversión. Estos modelos con ``ruidos'' pueden ser complejos de manejar ya que proveen predicciones estocásticas y no deterministicas \citeB{gabaix_behavioral_2019}. Estos ``ruido'' se pueden interpretar como señales que afectan las decisiones de inversión. Estos pueden clasificarse como señales de información erróneas sobre la inversión que pueden afectar el rendimiento del inversionista. Pero, se reconoce que en la realidad existe la posibilidad de gastar dinero en mejorar la información y por ende disminuir el ruido, gasto que se incluye en el modelo que se está evaluando.
\vspace{2.5mm}

Existe una parte de la información que es disponible, es definida en varios trabajos como información pública, de la cuál no se debe pagar y toda persona racional puede acceder. Luego, si se quiere aumentar la información beneficiosa y disminuir el ruido, se incurre en un costo asociado a obtener esa información (por ejemplo, el costo de contratar servicios de consultoría financiera). El trabajo \citeB{verrecchia_information_1982} fue de los primeros en explorar el incentivo de inversionistas de pagar para obtener información en los mercados financieros y mejorar sus inversiones. En otras palabras, demuestra que los agentes pagan para obtener mejor información y así obtener mejores señales y disminuir el ruido. 
\vspace{2.5mm}

\citeB{gabaix_behavioral_2019} propone un modelo de optimización donde se considera la precisión de la señal sujeta a su costo de mejorarla. En esta se explica el costo de inatención que tenemos como personas al tomar decisiones y su falta de representación en los modelos económicos.
\vspace{2.5mm}

Trabajos más contemporáneos proponen formas de caracterizar la forma en que estos costos de información toman en la realidad. Entender la forma de como caracterizarlos es necesario para modelar de forma exitosa su efecto. \citeB{dewan_estimating_2020} hace exactamente esto. En este trabajo, primero, se propone la función objetivo del tomador de decisión de la siguiente forma: 

$$\max_{P}\quad rP-C(P)$$

Donde $P \in [0,1]$ representa el \textit{performance} (rendimiento en castellano) elegido por el tomador de decisión. A mayor rendimiento, mayor será el total de \textit{reward} (premio en castellano) $r$ que el tomador obtendrá de la inversión o decisión tomada. $C(P)$ es el costo asociado a ese rendimiento. 
\vspace{2.5mm}

El modelado de este costo es lo que permite la resolución del problema del tomador de decisión. Este costo debe ser diferenciable\footnote{Una función diferenciable es aquella derivable por lo menos de primer orden} y bien comportada\footnote{\textit{well-behaved:} un costo continuo y convexo es bien comportado. \citeB{dewan_estimating_2020}}. Con esto, se llega a la \textbf{Proposición 4} del modelo en \citeB{dewan_estimating_2020} donde se demuestra que con el costo convexo, entonces la función $\max_{P}\, rP-C(P)$ es cóncava, por lo que existe un máximo local que es también global máximo que resuelve el modelo.
\vspace{2.5mm}

En \citeB{dewan_estimating_2020} se toma como ejemplo un costo $C(P)$ cuadrático, presentado en \ref{costocuad}, donde $d \in [0,1]$ representa la información libre y gratis para todos (información pública) y $c$ el costo marginal de información. 

\begin{equation}
\begin{array}{rrclcl}
    {\mathcal{C}}(P)=\begin{cases}0,&P\leq d\\c(P-d)^2,&P>d\end{cases}\label{costocuad}\\
\end{array}
\end{equation}
\vspace{2.5mm}

\ref{costocuad} muestra que a mientras el rendimiento sea mayor a la información pública, se incurrirá un costo de obtención de un \textit{performance} mayor posible gracias a la obtención de información adicional.
\vspace{2.5mm}

Finalmente, aplicando la \textbf{Proposición 4} se llega a que el rendimiento óptimo según el premio $r$ previamente definido, será dependiente de los costos asociados a la información y su magnitud en comparación con el premio. También, como estaba antes definido, $P$ no puede superar a 1 ya que no se puede obtener un premio mayor al ofrecido.

\begin{equation}
\begin{array}{rrclcl}
    P^*(r) = \begin{cases}\frac{r}{2c}+d,&r\leq 2c(1-d)\\1,&r>2c(1-d)\end{cases}, \label{perforopt}\\
\end{array}
\end{equation}





