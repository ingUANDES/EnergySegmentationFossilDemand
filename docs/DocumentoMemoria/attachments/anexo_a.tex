% anexo_a.tex

\chapter{Anexos}
\label{}



\section{Ejemplo Suma ponderada}\label{ej:sumapond}

Se tienen 3 escenarios posibles con $w = 1,2,3$ y parámetros $c_{w}= c_1,c_2,c_3$ y variables de diseño $x_{w}=x_{1}$, $x_{2}$,$x_{3}$. Luego, se tiene una suma ponderada de escenarios  $E_{\Theta}$[$x_{w}$ $c_{w}$] y la probabilidad de ocurrencia de cada escenario w es igual a $\Theta ={0.5 , 0.3 ,0.2 }$.
\vspace{2.5mm}

Con esto se tiene la siguiente expresión:

$$E_{\Theta} [ x_{w} c_{w}] = x_{1}c_{1}0.5 + x_{2}c_{2}0.3 + x_{3}c_{3}0.2 $$

\section{Parámetros Modelo Original}\label{anexo:parametros}


\section{Pruebas iniciales del rendimiento y atención racional en el subastador}\label{anexo:rendimiento}


Para realizar la evaluación del modelo, se realizará un gráfico con los resultados similar a la figura \ref{fig:fig6}. En esta se grafica la cantidad de permisos vendidos por el productor que produce energía con carbón y su influencia en el precio de mercado de los permisos. Esto representa el impacto que puede tener el modelo en eliminar el carbón como productor.

\begin{figure}[H]
    \centering
    \includegraphics[width=15cm]{docs/DocumentoMemoria/images/figura 6 amigo.png}
    \caption{Venta de permisos por generador de carbón (Fuente: \protect\citeB{amigo_two_2021})}
    \label{fig:fig6}
\end{figure}


\subsubsection{Modelo sin restricciones}

Primero se evaluará el modelo al considerar únicamente la función objetivo del subastador con el costo original $\mathcal{F}$ y el costo de rendimiento. También se considera como variable de decisión únicamente los permisos $\theta$.  

\begin{equation}
\begin{array}{rrclcl}
    \displaystyle \min_{\theta} & -\theta \pi^aP + \tilde{\mathcal{F}}(P)+F(\theta)  \label{fo:perfornorest}\\
\end{array}
\end{equation}
\begin{equation}
\begin{array}{cl}
    \theta \geq 0 & (\varrho)\label{res:sub2}
\end{array}
\end{equation}

Con lo anterior se logra el siguiente Lagrangiano:

$$\mathcal{L}(\theta)=-P\theta\pi^a+\mathcal{F}(\theta)+c(P-d)^2 - \varrho\theta $$

Realizando la derivada de primer orden para $\theta$ se obtiene:

\begin{equation}
\begin{array}{rrclcl}
    \frac{\partial\mathcal{L}(\theta)}{\partial (\theta)}=-P\pi^a+\frac{\partial\mathcal{F}(\theta)}{\partial(\theta)}-\varrho=0 \label{lag1}\\
\end{array}
\end{equation}
\begin{equation}
\begin{array}{rrclcl}
    \rightarrow -P\pi^a+\frac{\partial\mathcal{F}(\theta)}{\partial(\theta)}=\varrho \label{lag1}\\
\end{array}
\end{equation}
Con esto se obtiene la siguiente complementariedad para el problema del subastador:

\begin{equation}
\begin{array}{rrclcl}
    0\leq -P\pi^a+\frac{\partial\mathcal{F}(\theta)}{\partial(\theta)}\perp \theta \geq 0 \label{compllag1}\\
\end{array}
\end{equation}

Al igual que en el modelo original de \citeB{amigo_two_2021}, se consideran los siguientes valores para constantes y parámetros:
\begin{enumerate}
    \item $\frac{\partial\mathcal{F}(\theta)}{\partial(\theta)}=0$
    \item $CAP\sim \mathcal{N}(100MtCO_{2}e,0)$
\end{enumerate}

Con esto, al correr el modelo en el \textit{solver}, cambiando únicamente el valor del rendimiento ($P$), se encontraron los siguientes resultados:

\begin{table}[H]
\centering
\begin{tabular}{|l|l|l|}
\hline
\textbf{P($\%$)} & \textbf{$\theta$ (millones)} & \textbf{$\pi^a$($\frac{\$}{\theta}$)} \\ \hline
0 & N/A & N/A \\ \hline
5 & 104 & 223 \\ \hline
10 & 3203 & 0 \\ \hline
15 & 3203 & 0 \\ \hline
20 & 3203 & 0 \\ \hline
25 & 3203 & 0 \\ \hline
30 & 3203 & 0 \\ \hline
35 & 3203 & 0 \\ \hline
40 & 3203 & 0 \\ \hline
45 & 3203 & 0 \\ \hline
50 & 3203 & 0 \\ \hline
55 & 3203 & 0 \\ \hline
60 & 3203 & 0 \\ \hline
65 & 3203 & 0 \\ \hline
70 & 3203 & 0 \\ \hline
75 & 3203 & 0 \\ \hline
80 & 3203 & 0 \\ \hline
85 & 3203 & 0 \\ \hline
90 & 3203 & 0 \\ \hline
95 & 3203 & 0 \\ \hline
100 & 1870 & 0 \\ \hline
\end{tabular}
\caption{Resultados con rendimiento y sin restricciones}
\label{tabla:sinrestr}
\end{table}


Los resultados encontrados en el Cuadro \ref{tabla:sinrestr}, pueden ser resultado de no considerar una restricción para $\theta$ respecto al presupuesto de carbono en el problema del subastador. En la siguiente subsección se desarrolla el problema incluyendo la restricción adicional.

\subsubsection{Modelo con restricción original}

Manteniendo la restricción original de \citeB{amigo_two_2021} mostrada en \ref{res:sub1}. Se tiene el siguiente modelo:

\begin{equation}
\begin{array}{rrclcl}
   \displaystyle \min_{\theta} & -\theta \pi^aP + c(P-d)^2+F(\theta) \\\textrm{s.a.} \label{eq:perforconrestr}\\
\end{array}
\end{equation}
\begin{equation}
\begin{array}{cl}
    \varphi^-1 (\varepsilon )\sigma + \mu - \theta \geq 0 & (\eta)  \label{perforconrestr:r1}
\end{array}
\end{equation}
\begin{equation}
\begin{array}{cl}
    \theta \geq 0 & (\varrho)
\end{array}
\end{equation}

Al igual que el caso anterior, se debe convertir este modelo en un MCP. Para esto se aplican las condiciones de KKT.

$$\mathcal{L}(\theta)=-P\theta\pi^a+\mathcal{F}(\theta)+c(P-d)^2 -\eta(\varphi^-1 (\varepsilon )\sigma + \mu - \theta)- \varrho\theta $$

Realizando la derivada de primer orden para $\theta$ se obtiene:

\begin{equation}
\begin{array}{rrclcl}
    \frac{\partial\mathcal{L}(\theta)}{\partial (\theta)}=-P\pi^a+\frac{\partial\mathcal{F}(\theta)}{\partial(\theta)}-\eta -\varrho=0 \label{lag20}\\
\end{array}
\end{equation}
\begin{equation}
\begin{array}{rrclcl}
    \rightarrow -P\pi^a+\frac{\partial\mathcal{F}(\theta)}{\partial(\theta)}-\eta=\varrho \label{lag21}\\
\end{array}
\end{equation}

Obteniendo la primera complementariedad:

\begin{equation}
\begin{array}{rrclcl}
    0\leq -P\pi^a+\frac{\partial\mathcal{F}(\theta)}{\partial(\theta)}-\eta \perp \theta \geq 0 \label{compllag2}\\
\end{array}
\end{equation}

La segunda complementariedad se obtiene al considera la restricción \ref{perforconrestr:r1} con su variable dual respectiva $\eta$. Obteniendo:

\begin{equation}
\begin{array}{rrclcl}
    0 \leq \varphi^-1 (\varepsilon )\sigma + \mu - \theta \perp \eta \geq 0 \label{compllag2}\\
\end{array}
\end{equation}

Nuevamente, al igual que en el modelo original de \citeB{amigo_two_2021}, se consideran los siguientes valores para constantes y parámetros:
\begin{enumerate}
    \item $\frac{\partial\mathcal{F}(\theta)}{\partial(\theta)}=0$
    \item $CAP\sim \mathcal{N}(100MtCO_{2}e,0)$
\end{enumerate}

Con esto, al correr el modelo en el \textit{solver}, cambiando únicamente el valor del rendimiento ($P$), se encontraron los siguientes resultados:

\begin{table}[H]
\centering
\begin{tabular}{|l|l|l|}
\hline
\textbf{P($\%$)} & \textbf{$\theta$ (millones)} & \textbf{$\pi^a$($\frac{\$}{\theta}$)} \\ \hline
0 & 0.3336 & 30130 \\ \hline
5 & 100 & 315.825 \\ \hline
10 & 100 & 315.825 \\ \hline
15 & 100 & 315.825 \\ \hline
20 & 100 & 315.825 \\ \hline
25 & 100 & 315.825 \\ \hline
30 & 100 & 315.825 \\ \hline
35 & 100 & 315.825 \\ \hline
40 & 100 & 315.825 \\ \hline
45 & 100 & 315.825 \\ \hline
50 & 100 & 315.825 \\ \hline
55 & 100 & 315.825 \\ \hline
60 & 100 & 315.825 \\ \hline
65 & 100 & 315.825 \\ \hline
70 & 100 & 315.825 \\ \hline
75 & 100 & 315.825 \\ \hline
80 & 100 & 315.825 \\ \hline
85 & 100 & 315.825 \\ \hline
90 & 100 & 315.825 \\ \hline
95 & 100 & 315.825 \\ \hline
100 & 100 & 315.825 \\ \hline
\end{tabular}
\caption{Resultados con rendimiento y restricción original}
\label{tabla:conrestr}
\end{table}

Del Cuadro \ref{tabla:conrestr} se entiende que al incluir únicamente las restricciones del problema original solo existe un efecto en el precio y cantidad de permisos cuando el rendimiento es 0. Pero esto no debería ocurrir ya que por lo menos se tiene la información pública $d$. Cabe notar que una cantidad de $100 MtCO_2 e$ con un precio de 315.825 USD cada una, son los valores de resolución del problema original con los mismos parámetros. Este resultado puede resultar debido a que no se a incluido una restricción que describa de mejor forma el costo de rendir.
