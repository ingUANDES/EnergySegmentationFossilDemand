\chapter{Glosario}
\label{glossary}

\begin{itemize}
    \item \textbf{API}: API es una sigla que procede de la lengua inglesa y que alude a la expresión \textit{Application Programming Interface} (cuya traducción es Interfaz de Programación de Aplicaciones). El concepto hace referencia a los procesos, las funciones y los métodos que brinda una determinada biblioteca de programación a modo de capa de abstracción para que sea empleada por otro programa informático.
    \item \textbf{Core}: Hace referencia al núcleo computacional, el cual es un procesador individual al interior de la CPU.
    \item \textbf{Cluster}: El término clúster (del inglés \textit{cluster}, que significa grupo o racimo) se aplica a los conjuntos o conglomerados de ordenadores unidos entre sí normalmente por una red de alta velocidad y que se comportan como si fuesen una única computadora.
    \item \textbf{CPU}: La unidad central de procesamiento o unidad de procesamiento central (conocida por las siglas CPU, del inglés: \textit{central processing unit}), es el hardware dentro de un ordenador u otros dispositivos programables, que interpreta las instrucciones de un programa informático mediante la realización de las operaciones básicas aritméticas, lógicas y de entrada/salida del sistema.
    \item \textbf{DLB}: DLB son las siglas de Dynamic Load Balancing, el cual es el nombre que recibe la implementación del rebalanceo dinámico en esta memoria.
    \item \textbf{Framework}: En el desarrollo de software, un framework es una estructura conceptual y tecnológica de soporte definida, normalmente con artefactos o módulos de software concretos, en base a la cual otro proyecto de software  puede ser organizado y desarrollado. Típicamente, puede incluir soporte de programas, bibliotecas y un lenguaje interpretado  entre otros programas para ayudar a desarrollar y unir los diferentes componentes de un proyecto.
    \item \textbf{HPC}: La computación de alto rendimiento (\textit{High performance Computing} o HPC en inglés) es la agregación de potencia de cálculo para resolver problemas complejos en ciencia, ingeniería o gestión. Para lograr este objetivo, la computación de alto rendimiento se apoya en tecnologías computacionales como los clusters, los supercomputadores o la computación paralela. La mayoría de las ideas actuales de la computación distribuida se han basado en la computación de alto rendimiento.
    \item \textbf{Idleness}: Se dice que la CPU esta en estado \textit{idle} cuando no esta siendo utilizado por ningún programa. En este trabajo, se habla de idleness cuando la aplicación en un instante de tiempo dado esta esperando a los otros procesos y no esta utilizando sus recursos para contribuir al desarrollo de la simulación.
    \item \textbf{Infiniband}: InfiniBand es un bus de comunicaciones serie de alta velocidad, baja latencia y de baja sobrecarga de CPU, diseñado tanto para conexiones internas como externas. Sus especificaciones son desarrolladas y mantenidas por la Infiniband Trade Association (IBTA).
    \item \textbf{Malla}: Las mallas representan una aproximación a un dominio geométrico. Esta representará el grado de aproximación que se quiere lograr entre el modelo computacional y la realidad.
    \item \textbf{MPI}: MPI (\textit{Message Passing Interface}, Interfaz de Paso de Mensajes) es un estándar que define la sintaxis y la semántica de las funciones contenidas en una biblioteca de paso de mensajes diseñada para ser usada en programas que exploten la existencia de múltiples procesadores.
    \item \textbf{Nodo}: En esta memoria, la referencia a Nodo, hace alusión a un ordenador individual dentro de un clúster de cómputo.
    \item \textbf{Rank}: MPI permite crear grupos lógicos de procesos al interior de la aplicación donde en cada grupo, los procesos se distinguen entre ellos por medio de un identificador llamado rank. El rank siempre es relativo a un grupo.
    \item \textbf{SHM}: SHM son las siglas de Shared Memory, el cual es el nombre que este trabajo le dio a la creación de canales con \textit{buffers} de memoria compartida para la comunicación.
    \item \textbf{TLP}: TLP son las siglas de Two Level Partitioning, el cual es el nombre que este trabajo le dio a el particionamiento en dos niveles, que considera la información de la topología del cluster para el primer particionamiento y la cantidad de núcleos de cada nodo para el segundo particionamiento.
    \item \textbf{Topología}: En este trabajo, la topología representa el estudio de la geometría del conjunto de ordenadores comunicados entre sí para el intercambio de información, donde cada uno se denomina nodo. Además internamente, cada nodo tiene su topología, donde la información que se rescata de estos es la cantidad de núcleos computacionales que este posee.
\end{itemize}