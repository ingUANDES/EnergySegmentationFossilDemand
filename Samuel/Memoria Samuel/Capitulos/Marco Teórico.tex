% Marco Teórico.tex

\chapter{Marco Teórico}
\label{c2} 


\section{\textit{Cap and Trade} y otros sistemas actuales}\label{c21}
%Agregar desde segundo párrafo de la Revision bibliográfica del hito 1

En la actualidad, existen diversos sistemas y mecanismos implementados a nivel global para abordar el problema de las emisiones de gases de efecto invernadero y su impacto en el medio ambiente. Uno de estos sistemas es el \textit{Cap and Trade}, que establece límites máximos de emisiones para las empresas y permite el comercio de permisos de emisión entre ellas. Este enfoque busca incentivar la reducción de emisiones al otorgar beneficios económicos a las empresas que logren disminuir sus emisiones por debajo de los límites establecidos.\\

El concepto de \textit{cap and trade} se originó en la década de 1960, cuando el economista estadounidense Thomas Crocker propuso en su artículo la idea de utilizar el mercado para controlar la contaminación atmosférica \citeB{shogren_introduction_2011}. En su propuesta, sugirió establecer límites máximos de contaminantes y permitir que las empresas comercien con permisos de emisión dentro de esos límites.\\

Aunque ha habido diversas implementaciones y propuestas a lo largo de los años, se considera que el primer uso formal del modelo de \textit{Cap and Trade} fue en el marco de la Ley del Aire Limpio de Estados Unidos en 1990.\\

La Ley del Aire Limpio de Estados Unidos estableció el Programa de Comercio de Emisiones de Óxido de Azufre (SO2) como un medio para reducir las emisiones de dióxido de azufre, un contaminante que contribuye a la lluvia ácida \citeB{cahn_s02_2012}. Bajo este programa, se asignó un límite total de emisiones de SO2 (el \textit{tope} o \textit{cap}) a las instalaciones industriales y se les otorgaron créditos de emisión equivalentes a esa cantidad total. Las empresas podían comprar, vender o intercambiar estos créditos entre ellas, lo que permitía flexibilidad y fomentaba la reducción de emisiones de manera rentable.\\


Desde entonces, el modelo de \textit{Cap and Trade} ha sido utilizado en diversos contextos y países como una herramienta para abordar la contaminación y las emisiones de gases de efecto invernadero(GEI). Uno de los ejemplos más destacados es el Sistema de Comercio de Emisiones de la Unión Europea se han implementado varios sistemas de \textit{Cap and Trade} en diferentes países y regiones,este comenzó en 2005 y es el sistema de comercio de emisiones más grande del mundo\citeB{comision_europea_comunicacion_2019}.\\


Además de la Unión Europea, otros países y regiones han adoptado el modelo \textit{Cap and Trade} como estrategia para abordar el cambio climático. Un ejemplo destacado en Norteamérica es la implementación del Sistema de Comercio de Emisiones en California y Quebec, Canadá. Según \citeB{purdon_political_2014}, estos sistemas de \textit{Cap and Trade} no actúan de manera aislada, sino que forman parte de un conjunto más amplio de políticas diseñadas para combatir el cambio climático.\\

En ambas jurisdicciones, el sistema de \textit{Cap and Trade} se utiliza como una medida de apoyo para potenciar la eficacia de otros programas, estableciendo un precio al carbono. Esto incentiva la reducción de emisiones y promueve prácticas más sostenibles. Además, las políticas complementarias permiten a los gobiernos mantener un control significativo sobre la política climática, dirigiéndose a las fuentes de emisiones que suelen ser insensibles a las fluctuaciones de precios. De esta manera, el modelo de \textit{Cap and Trade} se convierte en una herramienta esencial dentro de un marco de políticas más amplio para abordar el cambio climático.\\

Aparte del modelo descrito, existen otros sistemas y políticas que buscan abordar el problema de las emisiones de gases de efecto invernadero. Por ejemplo, los impuestos al carbono, que gravan las emisiones de dióxido de carbono (CO2) y otros gases de efecto invernadero, tienen como objetivo desincentivar el uso de combustibles fósiles y promover la adopción de tecnologías más limpias y sostenibles.\\

El impuesto al carbono o impuesto a emisiones de CO2 es un mecanismo que implica un costo adicional para la generación de electricidad con energías no renovables, y se cobra por tonelada emitida de este compuesto. Según \citeB{asen_carbon_2021}, Finlandia fue el primer país en implementar este impuesto en 1990, y desde entonces, muchos otros países en todo el mundo han adoptado impuestos similares, aplicándolos a diferentes valores por tonelada emitida. En todo el mundo, se han implementado impuestos al carbono en más de 45 países hasta la fecha. Este mecanismo se ha convertido en una herramienta popular para incentivar la reducción de emisiones de gases de efecto invernadero y combatir el cambio climático.\\

En el caso de Chile, el gobierno ha presentado propuestas para combatir el cambio climático y reducir las emisiones de gases de efecto invernadero en el sector energético. Durante la Conferencia de las Naciones Unidas sobre el Cambio Climático (COP26) en octubre de 2021, el Ministerio del Medio Ambiente de Chile presentó una propuesta climática a largo plazo que incluye un límite de emisión de carbono para los generadores de energía (\citeB{gobierno_de_chile_estrategia_2021}). No obstante, la propuesta no detalla un mecanismo específico para que las empresas generadoras alcancen estos objetivos ni cómo se supervisará el cumplimiento de las metas establecidas.\\

En estos estudios, se destaca la relevancia de la implementación del modelo y cómo el presupuesto total de carbono puede influir en el mercado. No obstante, el propio creador del modelo de \textit{Cap and Trade}, Thomas Crocker, expresó en 2009 su escepticismo acerca de si un modelo de \textit{Cap and Trade} sería la forma más eficiente de regular el carbono, debido a la dificultad para predecir cómo este modelo afectaría al mercado de generación en el futuro. Además, si no se cuenta con una entidad reguladora que supervise las emisiones de manera constante y establezca las capacidades de forma adecuada, un modelo basado únicamente en impuestos a la emisión podría ser más eficaz.\\

Por lo tanto, un modelo que pueda regularse por sí mismo, en el que exista un incentivo para reducir las emisiones, en el que las empresas generadoras tengan más de una oportunidad para modificar sus inversiones en capacidad y permisos, y que este cambio represente un beneficio para las empresas, sería un enfoque adecuado para un modelo de \textit{Cap and Trade} efectivo. En este sentido, \citeB{amigo_two_2021} proponen un sistema que cumple con estas características.

\section{Evolución del sistema eléctrico en Chile}\label{c22}

El sistema eléctrico en Chile ha experimentado una serie de cambios significativos desde su diseño original hace cuatro décadas. Estos cambios han sido impulsados por una combinación de factores, incluyendo la liberalización del mercado, la introducción de nuevas tecnologías y la creciente preocupación por el medio ambiente.\\

En la década de 1980, Chile fue uno de los primeros países en el mundo en liberalizar su sector eléctrico. Esta liberalización se llevó a cabo en el marco de una serie de reformas económicas más amplias que buscaban reducir el papel del estado en la economía y promover la competencia y la eficiencia del mercado. Como resultado de estas reformas, se creó un mercado mayorista de electricidad en el que las empresas generadoras venden su electricidad a las empresas distribuidoras a través de un sistema de subastas competitivas.\\

En los primeros años del mercado mayorista, la generación de electricidad en Chile estaba dominada por un pequeño número de grandes centrales hidroeléctricas y térmicas. Sin embargo, a lo largo de los años, la matriz energética del país ha evolucionado para incluir una mayor diversidad de fuentes de energía. En particular, ha habido un crecimiento significativo en la generación de energía a partir de fuentes renovables, como la energía eólica y solar.\\

Este cambio en la matriz energética ha sido impulsado en parte por la introducción de políticas gubernamentales que promueven el uso de energías renovables. Por ejemplo, en 2008, el gobierno chileno introdujo una ley que requiere que una cierta proporción de la electricidad vendida por las empresas distribuidoras provenga de fuentes renovables. Esta proporción ha ido aumentando con el tiempo, y se espera que continúe aumentando en el futuro.\\

Además de la diversificación de la matriz energética, también ha habido cambios significativos en la estructura del mercado eléctrico. En particular, ha habido un movimiento hacia una mayor integración de los mercados eléctricos regionales. En 2017, los dos principales sistemas eléctricos de Chile, el Sistema Interconectado Central (SIC) y el Sistema Interconectado del Norte Grande (SING), se fusionaron para formar el Sistema Eléctrico Nacional (SEN). Esta fusión ha permitido una mayor eficiencia en la operación del sistema eléctrico y ha facilitado la integración de las energías renovables.\\

A pesar de estos cambios, el sistema eléctrico de Chile todavía enfrenta una serie de desafíos. Uno de los principales desafíos es la necesidad de equilibrar la creciente demanda de electricidad con la necesidad de reducir las emisiones de gases de efecto invernadero. Otro desafío es la necesidad de garantizar la seguridad del suministro de electricidad en un contexto de creciente variabilidad y incertidumbre en la generación de energía renovable.\\


Según \citeB{serra_chiles_2022}, algunos de los desafíos clave que enfrenta el sistema eléctrico de Chile incluyen:
\begin{enumerate}
\item 
La integración de las energías renovables: A medida que Chile aumenta su dependencia de las energías renovables, el sistema eléctrico debe adaptarse para manejar la variabilidad y la intermitencia asociadas con estas fuentes de energía. Esto puede requerir inversiones en infraestructura de red, así como el desarrollo de mercados de capacidad y servicios auxiliares para garantizar la estabilidad del sistema.
\item 
La descarbonización del sistema eléctrico: Chile se ha comprometido a reducir sus emisiones de gases de efecto invernadero y a descarbonizar su sistema eléctrico. Esto implica un cambio significativo en la matriz energética del país, con un alejamiento de los combustibles fósiles y un aumento en la generación de energía renovable. Este proceso de descarbonización también requerirá cambios en la regulación y la política energética.
\item 
La eficiencia del mercado eléctrico: Aunque el mercado eléctrico de Chile ha funcionado bien en general, existen áreas en las que se puede mejorar la eficiencia. Por ejemplo, el autor señala que el sistema actual de precios de la electricidad, que se basa en el costo marginal de la generación, puede no reflejar adecuadamente los costos totales de la generación de electricidad. Esto puede llevar a distorsiones en las señales de precio y a una asignación ineficiente de los recursos.
\item 
La equidad en el acceso a la electricidad: Aunque la cobertura eléctrica en Chile es alta, existen diferencias en el acceso a la electricidad entre las diferentes regiones y grupos socioeconómicos. Garantizar un acceso equitativo a la electricidad es un desafío importante, especialmente a medida que el país se mueve hacia una matriz energética más sostenible.
\end{enumerate}

Estos desafíos son complejos y requieren una combinación de soluciones técnicas, regulatorias y políticas. Sin embargo, a medida que Chile continúa su transición hacia una matriz energética más limpia y sostenible, estos desafíos también representan oportunidades para innovar y mejorar el sistema eléctrico del país.

\section{Benchmark sobre los precios de combustibles renovables y no renovables}\label{c23}

El costo del combustible es una consideración importante en cualquier aplicación, ya sea para impulsar un vehículo recreativo o proporcionar energía a una planta de fabricación.\\

\begin{table}[h]
\scriptsize 
\centering % Centra la tabla
\begin{tabular}{|c|c|c|>{\centering\arraybackslash}p{3cm}|c|}
\hline
\bf{Combustible} & \bf{Tipo}  & \bf{Precio(\$/gallon)} & \bf{Poder Calorífico Inferior (MJ/kg)}
 & \bf{Industrias en las que se utiliza}\\
\hline
Gasolina & No renovable & 2.96 & 44.4 & Transporte, generación eléctrica\\
\hline
Diésel & No renovable & 3.17 & 45.5 & Transporte, generación eléctrica, calefacción\\
\hline
Gas Natural (GNV) & No renovable & 2.11 & 53.9 & Transporte, generación eléctrica\\
\hline
Etanol & No renovable & 2.17 & 26.8 & Transporte, generación eléctrica\\
\hline
Kerosene & No renovable & 3.47 & 43.1 & Calefacción, aviación\\
\hline
Biodiésel & Renovable & 3.08 & 37.3 & Transporte, generación eléctrica\\
\hline
Bioetanol & Renovable & 2.47 & 21.1 & Transporte, generación eléctrica\\
\hline
Biogás & Renovable & 4.27\$/(MMBTU)* & 21.5 & Generación eléctrica, calefacción\\
\hline
Hidrógeno & Renovable & 4.3 (\$/kg) & 141.8 & Transporte, generación eléctrica\\
\hline
\end{tabular}
\begin{itemize}
    \item[a] Fuente: Precio disponible en: \url{https://afdc.energy.gov/fuels/prices.html}
\item[b] Fuente: Poder calorifico disponible en :\url{https://www.iea.org/data-and-statistics} y \url{http://www.fao.org/faostat/en/#data/QC}
\item[c] *Metric Million British Thermal Unit
\item[f] Tabla actualizada al 10/5/2023
\end{itemize}

\end{table}




En el caso del hidrógeno, su precio es medido en kilogramos en lugar de litros o galones.Para convertir el precio del hidrógeno verde de dólares por kilogramo a dólares por galón, es necesario conocer la densidad del hidrógeno y la conversión de unidades. La densidad del hidrógeno es de aproximadamente 0.08988 kg/m³, mientras que un galón equivale a aproximadamente 3.78541 litros. Por lo tanto, un galón de hidrógeno verde pesaría aproximadamente 0.339 kg (0.08988 kg/m³ x 3.78541 L/galón).\\

Dividiendo el precio por kilogramo (en este caso, 4.3 dólares) por la cantidad de kilogramos en un galón (0.339 kg), se obtiene un precio de aproximadamente 12.68 dólares por galón de hidrógeno verde.\\


En cuanto a los combustibles renovables que pueden reemplazar a los no renovables en diferentes industrias, aquí están algunas opciones:\\
\begin{enumerate}
\item 
Transporte: El biodiésel y el bioetanol pueden reemplazar al diésel y la gasolina en vehículos modificados o diseñados para funcionar con estos combustibles. El hidrógeno también es una opción para vehículos de pila de combustible.
\item 
Generación eléctrica: El biogás y el hidrógeno pueden ser utilizados en lugar de combustibles fósiles como el carbón, el gas natural y el diésel en plantas de generación eléctrica.
\item 
Calefacción: El biogás puede ser utilizado en lugar del gas natural y el kerosene para calefacción en hogares e industrias.
\item 
Industria: El biogás y el hidrógeno pueden reemplazar al gas natural y al carbón en procesos industriales que requieren calor o energía.\\
\end{enumerate}

Un combustible con un poder calorífico inferior (PCI) más alto se considera más eficiente en términos de aprovechamiento energético. El PCI se refiere a la cantidad de calor liberado cuando un combustible se quema por completo, excluyendo el calor latente del vapor de agua producido durante la combustión.\\

Cuando un combustible tiene un PCI más alto, significa que se libera más energía térmica por unidad de masa al quemarlo. Esto implica que se puede obtener más calor útil de dicho combustible en comparación con otros combustibles con un PCI más bajo.\\

El PCI es esencial para maximizar la eficiencia energética, reducir costos y promover la sostenibilidad en diversas industrias. En el transporte, un alto PCI significa mayor eficiencia y menor costo operativo, al ofrecer más energía por unidad de combustible. En la generación eléctrica, un alto PCI implica mayor producción de energía por combustible consumido, optimizando la eficiencia y los costos. Para la calefacción, un alto PCI garantiza una mayor cantidad de calor por combustible, siendo más eficiente y económico.\\

El gas natural es una fuente de energía fósil, pero es más limpia que la gasolina y el diésel, y su precio es más estable que el de los combustibles líquidos. Además, el gas natural vehicular (GNV) puede ser una fuente de energía renovable si se produce a partir de fuentes renovables, como el biogás.\\

Por otro lado, el hidrógeno es un combustible renovable que se puede producir a partir de fuentes de energía renovable, como la energía solar y eólica. Aunque el hidrógeno es más caro que los combustibles fósiles, su precio está disminuyendo a medida que se desarrollan nuevas tecnologías para su producción y almacenamiento. Se puede observar en la tabla que el poder calorífico inferior del hidrógeno es, por lejos, el más alto.\\

\section{Amigo, Cea-Echenique \& Feijoo (2021): Un enfoque de \textit{cap and trade}}\label{c24}

Como se discutió en el Capítulo \ref{c21}, los sistemas de \textit{cap and trade}, al menos en teoría, pueden funcionar eficazmente cuando se aplica el mecanismo adecuado para el problema que se quiere solucionar. En este contexto, el modelo de \textit{cap and trade} propuesto por \citeB{amigo_two_2021} se destaca por su enfoque en la reducción de las emisiones de $CO_2$ y la promoción de la transición hacia fuentes de energía más sostenibles. Este modelo reconoce que tanto los productores como el planificador social (subastador o regulador) buscan maximizar sus beneficios.\\

El modelo propone un sistema de dos etapas que involucra a los generadores de electricidad y al subastador que proporciona los permisos de emisiones de dióxido de carbono ($CO_2 e$). En la primera etapa (periodo $t=0$), cada productor de electricidad $i$ deciden su cantidad de generación $Q_i(t)\in\mathbb{R}_+$ de manera tal que maximice sus utilidades (ver ~\ref{eq:revenew}), la capacidad instalada $\bar{Q}_i$ y la de asignación de sus permisos de emisión de carbono $A_i\in\mathbb{R}_+$ que le compran a un subastador o agente regulador, en un precio de $\pi^{a}\in\mathbb{R}_+$  para satisfacer una demanda exógena $D(0)\in\mathbb{R}_+$. El precio con el que cada productor es pagado se define con $\pi^d(0)\in\mathbb{R}_+$. Por ultimo, cada productor $i$ decide la capacidad adicional $x_i(t)\in\mathbb{R}_+$ con un gasto de capital de $I_i\in\mathbb{R}_+$, esta capacidad adicional esta disponible después de un tiempo de construcción predefinido medido en años.\\

En la segunda etapa ($t\in T:={1,...\bar{t}}$), la incertidumbre se revela en el período $t=1$. La incertidumbre se representa mediante un estado de la naturaleza $\omega\in\Omega:={1,..,K}$ con una probabilidad definida por $Pr(\omega)\in[0,1]$. Por lo tanto, para un estado de la naturaleza dado $\omega\in\Omega$, el par $(t,\omega)$ representa la demanda en el período $t>1$ cuando se alcanza el estado $\omega$. Para cada par $(t,\omega)$, dado un asignación inicial de asignaciones de emisiones CO$2e$ en el período $t=0$ y un comercio secundario de permisos en $t=1$, los generadores eléctricos participan en un mercado spot y deciden su generación y nuevas inversiones de capacidad de manera que se satisfaga el nivel de demanda estocástica exógena $D(t,\omega)\in\mathbb{R}+^{T\times\Omega}$. La demanda para todos los períodos se denota como $D=\left(D(0),(D(t,\omega)){(t,\omega)\in T\times\Omega}\right)\in\mathbb{R}+\times\mathbb{R}_+^{T\times\Omega}$.\\

En esta misma etapa, para cada par $(t,\omega)$, el productor $i$ maximiza su beneficio eligiendo el nivel de generación $Q_i(t,\omega)\in\mathbb{R}+$ a un precio $\pi^d(t,\omega)\in\mathbb{R}+$. Consideramos un cambio tecnológico $TC_i(t) \in\mathbb{R}+$ que ajusta el costo marginal de diferentes tecnologías. De manera similar, cada productor $i$ elige una capacidad adicional $x_i(t,\omega)\in\mathbb{R}+$ con un costo de inversión de $TCR_i(t)\cdot I_i$, donde $TCR_i(t)\in\mathbb{R}+$ es el cambio en el costo de inversión o gasto de capital a lo largo del tiempo. Además, consideramos un sistema de comercio de permisos en $t=1$, donde los productores pueden comprar $P_i(\omega)\in\mathbb{R}+$ permisos de otros productores si necesitan superar la asignación inicial de permisos $A_i$, o vender $V_i(\omega)\in\mathbb{R}+$ permisos no utilizados. El precio de la transacción neta en este sistema de comercio está dado por $\pi^v(\omega)\in\mathbb{R}+$.\\


\section{Parámetros función del productor: Amigo, Cea-Echenique \& Feijoo (2021)}\label{c25}

Como ya vimos entonces en \ref{c24}, 
 los productores se enfrentan a la tarea de ajustar sus niveles de generación de energía y su inversión en permisos de emisión en respuesta a la demanda revelada. Estos deben tomar la decision sobre cuánta energía generar y cuántos permisos de emisión comprar o vender, teniendo en cuenta la demanda revelada y los precios de los permisos de emisión.\\

 Cada productor $i$ representa una única tecnología en la economía (ver Sección ...). El conjunto de variables del productor $i$ en un horizonte temporal de $\bar{t}$ años se compone de: 
 \begin{enumerate}

\item 
Una cantidad de expansión de capacidad $x_i:=\left(x_i(0),(x_i(t,\omega)){(t,\omega)\in T\times\Omega}\right)\in\mathbb{R}+\times\mathbb{R}+^{T\times\Omega}$
 \item 
 Un plan de producción $Q_i:=\left(Q{i}(0),(Q_{i}(t,\omega){(t,\omega)\in T\times\Omega})\right)\in\mathbb{R}+\times\mathbb{R}+^{T\times\Omega}$
 \item 
 Asignaciones de emisiones $A_i\in\mathbb{R}+$ compradas en el primer período $t=0$
 \item 
 Permisos $P_i(\omega)\in\mathbb{R}+^{\Omega}$ comprados en $t=1$ para el intervalo de tiempo $t\in[1,\bar{t}]$
 \item 
 Permisos $V_i(\omega)\in\mathbb{R}+^{\Omega}$ vendidos en $t=1$ para el intervalo de tiempo $t\in[1,\bar{t}]$.

\end{enumerate}
El modelo se ejecuta en intervalos de un año, por lo tanto, representamos el año en bloques de $\tau=8760$ horas. Para cada tecnología, consideramos un factor de capacidad $CF_i\in\mathbb{R}_+$ que representa la operación real de cada planta.\\

El objetivo de los productores es minimizar el costo. Por lo tanto, para un productor representativo, hay un costo de producción para cada período $t\in T$. Definimos la función de ingresos dada por los precios de la electricidad $\pi^d:=\left(\pi^d(0),\left(\pi^d(t,\omega)\right){(t,\omega)\in T\times\Omega}\right)\in\mathbb{R}+\times\mathbb{R}+^{T\times\Omega}$ (denotada como $p$ en la forma general descrita en la ecuación \ref{eq:revenew}) y los parámetros $(a_i,b_i){i\in{1,...N}}\in(\mathbb{R}^2_+)^N$ mediante

\begin{align}\label{eq:revenew}
f_i(p,q)=\Big(a_i\cdot q+\frac{b_i}{2}\cdot q^{2}\Big)-p\cdot q.
\end{align}\\

Definiendo $T_0:={0}\cup T$, el problema de optimización del productor $i$ se formula eligiendo \linebreak$(x_i,Q_i, A_i,P_i,V_i)\in\mathbb{X}:=\left(\mathbb{R}_+\times\mathbb{R}_+^{T\times\Omega}\right) \times\left(\mathbb{R}_+\times\mathbb{R}_+^{T\times\Omega}\right) \times \mathbb{R}_+\times\mathbb{R}_+^{\Omega}\times\mathbb{R}_+^{\Omega}$ dados los precios \linebreak$(\pi^d$, $\pi^a$ , $\pi^v)\in\Pi:=\left(\mathbb{R}_+\times\mathbb{R}_+^{T\times\Omega}\right)\times\mathbb{R}_+\times\mathbb{R}^{\Omega}_+$ , los parámetros \linebreak$\left((a_i,b_i),I_i, TC_i(t,\omega), TCR_i(t,\omega), CF_i,\bar{Q}_i, RP_i , \varepsilon_i\right)\in \Xi:=\mathbb{R}_+^2\times\mathbb{R}_+^7$, $\tau\in \mathbb{R}_+$ y la probabilidad $(Pr(\omega))_{\omega\in\Omega}\in\Delta:=\left\{\left(Pr(\omega)\right)_{\omega\in\Omega}\in[0,1]^K:\sum_{\omega\in\Omega}Pr(\omega)=1\right\}$ como una solución de la ecuación del problema de los productores \ref{fo:prod}. Además este va a estar sujeto a distintas restricciones; de capacidad en generación en la restricción \ref{res:1}, restricción en generación inicial en \ref{res:2}, de límites en capacidad en \ref{res:3} y venta y compra de permisos en \ref{res:4} y \ref{res:5}.


\begin{align}
\min_{(x_i,Q_i,A_i,P_i,V_i)\in \mathbb{X}} & f_i \big( \pi^d(0),Q_i(0)\big)+ A_i \pi^{a} + I_i x_i(0) \nonumber \\ 
& + \sum_{\omega} Pr(\omega)   \Bigg[ \sum_{t>0} \frac{1}{(1+R)^t} \Big[ TC_i(t,\omega)\cdot f_i \big( \pi^d(t,\omega),Q_i(t,\omega) \big) \nonumber \\
& + TCR_i(t,\omega) \cdot I_i\cdot x_i(t,\omega) \Big] + \pi^v(\omega)\cdot \big(P_i(\omega)-V_i(\omega)\big) \Bigg] \label{fo:prod} \\
\textrm{s.t \ } \nonumber
\end{align}
\begin{align}
\Big(CF_i \cdot\tau\Big)  \Bigg[\bar{Q}_i + \sum_{t^{\prime}<\bar{t}} x_i(t^\prime,\omega) + x_i(0)+ \bar{Q}_i(t) \Bigg] - Q_i(t,\omega) & \geq 0  & \forall  \quad i,\omega, t  > 0 & \quad (\alpha_{i,\omega,t})\label{res:1} \\
\Big(CF_i\cdot\tau \Big)\bar{Q_i}-Q_{i}(0) & \geq 0  & \forall  \quad i & \quad (\kappa_i) \label{res:2} \\
RP_i - \bar{Q}_i  - x_i(0) - \sum_{t > 0} x_i(t,\omega) & \geq 0 &  \forall \quad i,\omega &   \quad (\psi_{i,\omega}) \label{res:3} \\
A_{i} -V_i(\omega) & \geq  0  & \forall  \quad \omega & \quad (\beta_{i,\omega}) \label{res:4} \\
A_{i} + (P_i(\omega) - V_i(\omega))-\sum_{t>0}Q_i(t, \omega)\cdot \varepsilon_{i}-Q_i(0)\varepsilon_{i} & \geq  0  &\forall \quad \omega & \quad (\gamma_{i,\omega})\label{res:5} \\
Q_i(0) & \geq  0 & \forall \quad i & \quad (\lambda_i) \label{res:q0} \\ 
Q_i(t, \omega) & \geq  0   & \forall \quad \omega, t >0 & \quad (\delta_{i,\omega,t})\label{res:qt} \\
x_i(0) & \geq  0 & \forall  \quad i & \quad (\xi_i)  \label{res:capi0} \\ 
x_i(t, \omega) & \geq  0   & \forall  \quad \omega, t >0 & \quad (\varphi_{i,\omega,t})\label{res:capt}
\end{align}


El modelo de \citeB{amigo_two_2021} utiliza una serie de parámetros en su función del productor\ref{fo:prod}. Aquí se presenta una descripción de los más relevantes que se podrían considerar para evaluar la viabilidad económica de la transición hacia combustibles renovables:
\begin{enumerate}

\item 
$I_i$(USD/MW): Costo de expansión por tecnología i. Este parámetro es crucial para evaluar los costos de inversión necesarios para aumentar la capacidad de producción de una tecnología de energía específica.
\item 
$Q_i$(MW): Capacidad de operación actual por tecnología i. Este variable permite evaluar la capacidad de producción actual de diferentes tecnologías de energía.
\item 
$C_i$(USD/MWh): Costo de operación por tecnología i. Este parámetro es esencial para entender los costos operativos asociados con cada tecnología de energía.
\item 

$\varepsilon_{i}$ (tCO2e/MWh): Factor de emisión de la tecnología i. Este parámetro permite cuantificar las emisiones de CO2 asociadas con cada tecnología de energía, lo cual es crucial para evaluar el impacto ambiental de la transición hacia combustibles renovables.
\item 
$TC_i(t)$: Cambio en el costo de operación por tecnología. Este parámetro permite evaluar cómo los costos operativos de una tecnología de energía específica podrían cambiar con el tiempo.
\item 
$TCR_i(t)$: Cambio en el costo de inversión por tecnología. Este parámetro permite evaluar cómo los costos de inversión de una tecnología de energía específica podrían cambiar con el tiempo.
\item 
$CF_i$: Factor de capacidad por tecnología. Este parámetro permite evaluar la eficiencia de diferentes tecnologías de energía.
\item 
$RP_i$ (MW): Potencial de recurso por tecnología. Este parámetro permite evaluar el potencial de producción de diferentes tecnologías de energía.
\end{enumerate}

Estos parámetros van a permitir evaluar tanto los costos económicos como los impactos ambientales de la transición hacia combustibles renovables.