% Intro.tex

\chapter{Introducción}
\label{c1} % la etiqueta para referencias




El European Green Deal es una iniciativa de la Unión Europea que busca transformar su economía en una más sostenible y neutral en carbono. El objetivo mas próximo de esta es reducir las emisiones de gases de efecto invernadero en al menos un 55\% para el año 2030 en comparación con los niveles de 1990 \citeB{comision_europea_comunicacion_2019}\\

Para lograr este objetivo, establecieron una serie de medidas y políticas en diferentes sectores, incluyendo la industria del transporte y la producción de energía. En el sector del transporte, se promueve la electrificación de los vehículos y la reducción de las emisiones de los vehículos de combustión interna. Además, se busca fomentar el uso del transporte público y la movilidad sostenible.\\

En el sector de la producción de energía, el European Green Deal busca promover la transición hacia fuentes de energía renovable y la eliminación gradual de los combustibles fósiles. Para lograr este objetivo, se están implementando medidas para reducir la dependencia de los combustibles fósiles y promover el uso de fuentes de energía renovable, cómo la energía eólica y solar, o los distintos combustibles renovables cómo por ejemplo el hidrógeno verde.\\

La Estrategia Europea del Hidrógeno es una de las principales políticas que se desarrollan en el marco del pacto. Se están invirtiendo en tecnologías de producción de hidrógeno verde más eficientes y se están estableciendo incentivos para su uso en diferentes sectores, como el transporte, la producción de energía y las industrias en general. Para esto existe una estrategia que establece los principales campos de actuación para desarrollar el papel del hidrógeno limpio en la reducción de emisiones de la economía de la UE de manera eficiente. La Estrategia establece tres horizontes temporales (2024, 2030 y 2050) para los que determina una sucesión de hitos a alcanzar.\\

En la primera fase (2020-2024), se instalarán al menos 6 GW de electrolizadores en la UE y se producirá hasta 1 millón de toneladas de hidrógeno renovable. En la segunda fase (2025-2030), se espera instalar al menos 40 GW de electrolizadores para 2030 y producir hasta 10 millones de toneladas de hidrógeno renovable en la UE. En la tercera fase (2030-2050), se espera que las tecnologías de hidrógeno renovable alcancen la madurez y se desplieguen a gran escala para llegar a todos los sectores difíciles de descarbonizar.\\

La Estrategia Europea del Hidrógeno establece que el hidrógeno renovable es esencial para respaldar el compromiso de la UE de alcanzar la neutralidad de carbono en 2050 y para respaldar el esfuerzo global para implementar el Acuerdo de París. Además, indica que el ecosistema del hidrógeno en Europa es probable que se desarrolle a través de una trayectoria gradual, a diferentes velocidades en los diferentes sectores y posiblemente en las distintas regiones, requiriendo, por tanto, múltiples soluciones políticas.\\




En este contexto, la transición hacia fuentes de energía renovable se ha convertido en una alternativa cada vez más atractiva y necesaria. En Chile, la producción y consumo de energía son fundamentales para el desarrollo económico y social del país, por lo que es crucial evaluar la viabilidad económica de la transición hacia combustibles renovables.\\

La transición hacia fuentes de energía renovable es la única forma de garantizar un futuro energético sostenible y seguro para Chile. Actualmente, Chile tiene una gran dependencia de los combustibles fósiles, que representan alrededor del 70\% de la matriz energética chilena \citeB{bernal_matriz_nodate}. Además, Chile es vulnerable a los efectos del cambio climático, lo que subraya la importancia de reducir las emisiones de gases de efecto invernadero.\\

Por lo tanto, la transición hacia fuentes de energía renovable es crucial para Chile, no solo para reducir su huella de carbono, sino también para mejorar la seguridad energética y fomentar el desarrollo sostenible. Esta transición es una oportunidad para Chile de liderar la transición energética en América Latina. Sin embargo, esta transición no es sencilla y requiere una planificación cuidadosa, la inversión en tecnologías y la implementación de políticas efectivas.\\

En este sentido, la Corporación de Fomento de la Producción (CORFO) tiene sus principios apuntando en la misma dirección. José Miguel Benavente el Vicepresidente ejecutivo de la CORFO comenta que se ha creado un comité de esta organización. En esta se tienen una nueva estrategia nacional del compromiso de descarbonización, en la que se vincula a los temas industriales del hidrógeno verde en Chile. Este comité lo lidera el Ministerio de Energía.\\

Chile acaba de firmar un acuerdo para que de aquí a 3 años más seamos la mayor producción de hidrógeno verde del mundo. Actualmente la CORFO está atrayendo inversionistas para realizar esta transición a producir hidrógeno verde. Esto lo hace de dos maneras; transformar la industria local para incorporar hidrógeno verde y la otra tiene que ver con la oferta, la parte productiva de electrolizadores.\\

Pretende adelantarse en la carrera de la producción de electrolizadores, comenta que "Podemos ser marginalmente más caros, incluso corrigiendo por los costos de transporte, pero si tenemos una energía que es más barata en términos relativos, energía verde, obviamente hay una compensación entre uno y otro y por lo tanto eso no se transmite al precio final del hidrógeno verde"\citeB{benavente_hidrogeno_2023}.\\

Con esto podemos concluir que aunque el costo de producción del hidrógeno verde en Chile podría ser ligeramente más alto que en otros lugares, esto podría ser compensado por el hecho de que la energía utilizada para producirlo es más barata en términos relativos, ya que proviene de fuentes de energía renovable y verde. Por lo tanto, el precio final del hidrógeno verde no se vería afectado por el costo de producción marginalmente más alto, ya que la energía utilizada para producirlo es más barata y compensa ese costo adicional. Todo esto permitiría al país tener una mayor independencia energética.\\

Además, existen diversos estudios que demuestran el potencial de Chile para la producción de biocombustibles (BC) mediante diferentes métodos. Un ejemplo de esto es un estudio llevado a cabo por la Universidad Técnica Federico Santa María \citeB{ortega_evaluacion_2010}, el cual señala que la opción más viable para producir BC en Chile es el bioetanol a partir del maíz o biodiesel utilizando raps. Sin embargo, es importante mencionar que esta alternativa implicaría un costo superior al de importar directamente los biocombustibles. Pero al igual que lo ya mencionado anteriormente, estaría brindaría al país la ventaja de lograr una mayor diversificación en su matriz energética.\\

 También tenemos el informe \citeB{ortega_evaluacion_2007}, este también concluye que es factible la producción de biodiesel  a partir de aceites vegetales de primer uso y el bioetanol a partir de diferentes cultivos, como la caña de azúcar, la remolacha y el maíz respectivamente.\\


Considerando el enorme potencial que Chile tiene para la producción de biocombustibles y el fuerte respaldo del gobierno, es probable que las diversas industrias se sientan motivadas a realizar una transición hacia los combustibles renovables. Los productores de energía tendrán la opción de generar energía limpia, mientras que la industria del transporte podrá utilizar combustibles limpios, y lo mismo sucederá en todas las demás industrias.\\

\section{Motivación}
Aunque la transición hacia fuentes de energía renovable se presenta como una oportunidad de gran valor para el país, también plantea numerosos desafíos. Los costos de inversión iniciales son altos y las tecnologías aún están en desarrollo. Además, la producción y utilización de biocombustibles y el hidrógeno verde implican cuestiones técnicas y regulatorias que deben ser abordadas.\\

El costo de los combustibles fósiles y sus consecuencias ambientales, unidos a la necesidad de reducir la dependencia energética y de cumplir con los objetivos de descarbonización, hacen que el desarrollo de los combustibles renovables sea imprescindible. Sin embargo, para poder implementar eficazmente esta transición, es necesario llevar a cabo un análisis en profundidad de los factores económicos, técnicos y regulatorios involucrados.\\

Este trabajo tiene como objetivo proporcionar un análisis integral de la viabilidad económica de la transición hacia combustibles renovables en Chile. Con este estudio se pretende proporcionar una herramienta que permita a los encargados de tomar decisiones evaluar las implicaciones económicas de la transición hacia los combustibles renovables, y así desarrollar políticas y estrategias eficaces para fomentar su uso.\\

Estoy convencido de que es necesario trabajar en soluciones sostenibles con el medio ambiente, y esta memoria es mi pequeña contribución en ese sentido. Espero que los hallazgos y conclusiones de este trabajo puedan servir como base para la implementación de políticas y acciones concretas en favor del medio ambiente y el desarrollo sostenible de Chile.\\


\section{Objetivos}
\subsection{Objetivo general}
El objetivo general del presente trabajo consiste en evaluar la viabilidad económica de la transición hacia combustibles renovables en el marco teórico del modelo de \textit{cap and trade} de dos etapas con multiples productores, creado por \citeB{amigo_two_2021}.\\

\subsection{Objetivos específicos}

\begin{enumerate}
\item 
\item 
\item 
\item 
\item 
\item 
\end{enumerate}


\section{Alcances}




\section{Metodología}



\newpage
\section{Estructura del documento}

La estructura del documento se divide en cinco capítulos: introducción, marco teórico, desarrollo metodológico, análisis de resultados y conclusiones.


