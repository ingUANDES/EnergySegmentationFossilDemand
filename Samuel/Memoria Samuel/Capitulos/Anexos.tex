% anexo_a.tex

\chapter{Anexos}
\label{}

\section{Códigos Resolución Maere 3.3.1}

\subsection{Código Resolución con Python}\label{codigoPhyton}
\begin{footnotesize}
\begin{lstlisting}
from gurobipy import Model, GRB, quicksum

def modeloRNcompetitivecapacityequilibrium(cant_esce, a, theta, B, tau, c, I):
    m = Model()

    # Decision variables x, Y, Q
    x = m.addVar(lb=0, vtype=GRB.CONTINUOUS, name='x')  # Capacidad
    Y = {}  # Producción
    Q = {}  # Demanda

    # Definición de Producción y Demanda para cada estado de la naturaleza
    for i in range(cant_esce):
        Y[i] = m.addVar(vtype=GRB.INTEGER, name='Y{}'.format(i))
        Q[i] = m.addVar(vtype=GRB.INTEGER, name='Q{}'.format(i))

    # Restricciones
    for i in range(cant_esce):
        # Production equals demand
        m.addConstr(Y[i] == Q[i])
        # Production bounded by installed capacity
        m.addConstr(Y[i] <= x)

    # F.O
    obj_expr = I * 1000 * x + tau * quicksum(theta[i] * (c * Y[i] - a[i] * Q[i] 
    + (B / 2) * (Q[i] * Q[i])) for i in range(cant_esce))
    m.setObjective(obj_expr, GRB.MINIMIZE)

    m.optimize()

    P = []
    IM = []

    for i in range(cant_esce):
        P.append(a[i] - B * Q[i].x)
        IM.append((tau * (P[i] - c) * Y[i].x - I * x.x) / x.x)

    Qprom = sum(Q[i].x for i in range(cant_esce))

    print("Q Promedio =", Qprom / cant_esce)
    print("x =", x.x)
    print("Valor función objetivo =", m.objVal)

    print(cant_esce, a, theta, B, tau, c, I)

    print("\nQ, Y, P and Investment Margin values:")
    for i in range(cant_esce):
        print("Q({}) = {:.1f}    Y({}) = {:.1f}    P({}) = {:.1f}    
        Investment Margin = {:.1f}"
          .format(i + 1, Q[i].x, i + 1, Y[i].x, i + 1, P[i], IM[i]))
\end{lstlisting}
\end{footnotesize}

\subsection{Código Resolución con Julia}\label{codigoJulia}
\begin{footnotesize}
\begin{lstlisting}
using JuMP, Ipopt
A = [300, 350, 400, 450, 500]
theta = [0.2, 0.2, 0.2, 0.2, 0.2]
B = 1
I = 90
C = 60
tau = 8760

model = Model(Ipopt.Optimizer)

# Variables
@variable(model, x >= 0)  # Capacidad
@variable(model, Y[1:5] >= 0) # Producción en escenario w
@variable(model, Q[1:5] >= 0) # Demanda en escenario w

# Restricciones
@constraint(model, demanda[i = 1:5], Y[i] == Q[i])
@constraint(model, capacidad[i = 1:5], Y[i] - x <= 0)

# Función Objetivo
@objective(model, Min, I*1000*x + tau * sum(theta[i] * (C * Y[i] - A[i] * 
Q[i] + B * 0.5 * Q[i]^2) for i = 1:5))
optimize!(model)

# Resultados
println("Q = ", value.(Q))
Qlist = value.(Q)

j = 1
P = []
while j < 6
    Pi = A[j] - B * Qlist[j]
    append!(P, Pi)
    j = j + 1
end
println("P = ", P)
println("x = ", value(x))
\end{lstlisting}
\end{footnotesize}